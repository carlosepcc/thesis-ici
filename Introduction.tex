\introduction
El campo de desarrollo que proyecta el escenario de la informaci�n y las comunicaciones en los �ltimos tiempos ha demostrado ser un catalizador por excelencia en el inicio del siglo XXI, marcando huellas en la Industria del Software que conducen a perfeccionar espacios educativos, did�cticos y cient�ficos, estableciendo dis�miles tendencias. Modernizar el espacio en que el hombre vive y convive es una de las proyecciones que caracteriza la revoluci�n cient�fica concentrando su actividad fundamental en la fabricaci�n de software, que hace m�s placentera, exitosa y prometedora las m�ltiples tareas a las que convoca enfrentar.
\\
\\
El hombre, en cualquier parte del mundo, se rige por reglas y normativas asignadas al control de la
sociedad, para poner freno a todas las infracciones cometidas por este. Desde los comienzos de la
humanidad fue necesario el reconocimiento de un conjunto de leyes que regularan ciertas manifestaciones del hombre en el entorno. La violaci�n de estos c�digos trae consigo una infracci�n por la que el Estado como garante o las instituciones representantes deben hacer cumplir. \citep{acanda2018cdis}
\\
\\
En Cuba, la Constituci�n de la Rep�blica de Cuba de 19871, establece las pautas fundamentales a seguir por todos los cubanos. Las instituciones en el pa�s se administran haciendo cumplir un conjunto de reglas y pol�ticas determinadas para su accionar; donde espec�ficamente en el caso del \ac{mes} , se rige por el reglamento disciplinario para los estudiantes de la educaci�n superior, puesto en vigor mediante la Resoluci�n No. 240 del a�o 20073.
La \ac{uci}, ya que pertenece al \ac{mes}, est� sujeta a cumplir este reglamento, lo cual conlleva a que si un estudiante o trabajador de la entidad incumple con alg�n o algunos de los art�culos descritos en su cuerpo legal, este debe ser procesado por una comisi�n disciplinaria en su car�cter de �rgano designado por una facultad para procesar a un estudiante o trabajador sancionado.
La gesti�n de las comisiones disciplinarias en una facultad docente posee una gran importancia debido a sus implicaciones legales