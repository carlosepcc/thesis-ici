\section{Proceso de comisi�n disciplinaria en la Facultad 4}\label{process}
%TODO
%Descripci�n del proceso de comisi�n disciplinaria en la Facultad 4
El proceso de comisi�n disciplinaria comienza cuando un profesor o estudiante de la \uci presenta una denuncia ante la Direcci�n de la Facultad. La Direcci�n de la Facultad es la encargada de verificar la veracidad de la denuncia y de asignar una comisi�n disciplinaria para que investigue el caso. La comisi�n disciplinaria es la encargada de realizar la investigaci�n y de emitir una resoluci�n. La resoluci�n puede ser de tres tipos: \textit{absoluci�n}, \textit{sanci�n} o \textit{no procede}. La resoluci�n es emitida por la comisi�n disciplinaria y es revisada por la Direcci�n de la Facultad. La Direcci�n de la Facultad es la encargada de notificar al profesor o estudiante acerca de la resoluci�n emitida por la comisi�n disciplinaria. 

% La resoluci�n puede ser apelada ante la Direcci�n de la Facultad. La Direcci�n de la Facultad es la encargada de revisar la apelaci�n y de emitir una resoluci�n final.

