\section{Pruebas}
Uno de los pilares de \ac{xp} es el proceso de pruebas. Esta metodolog�a de desarrollo anima a probar constantemente tanto como sea posible. Esto permite aumentar la calidad de los sistemas reduciendo el n�mero de errores no detectados y disminuyendo el tiempo transcurrido entre la aparici�n de un error y su detecci�n. Tambi�n permite aumentar la seguridad de evitar efectos colaterales no deseados a la hora de realizar  dise�ada por los programadores, y pruebas de aceptaci�n o pruebas funcionales destinadas a evaluar si al final de una iteraci�n se consigui� la funcionalidad requerida dise�adas por el cliente final \citep{gutierrez2006pruebas}.
Con el objetivo de comprobar que los sistemas desarrollados funcionan de acuerdo a las especificaciones descritas por el cliente, se realizaron diferentes pruebas teniendo en cuenta las caracter�sticas de los m�dulos.
 
\subsection{Pruebas unitarias}
Las pruebas unitarias o unittesting son una forma de comprobar que un fragmento de c�digo funciona correctamente.Son peque�os tests que validan el comportamiento de un objeto y la l�gica \citep{gutierrez2006pruebas}. \ac{xp} plantea la realizaci�n de pruebas unitarias continuas, frecuentemente repetidas y automatizadas, incluyendo pruebas de regresi�n, y aconseja escribir el c�digo de la prueba antes de la codificaci�n \citep{kniberg2007scrum}.\\
Para la aplicaci�n de las pruebas unitarias a la soluci�n se emple� el framework JUnit el cual permite realizar la ejecuci�n de clases Java de manera controlada, para poder evaluar si el funcionamiento de cada uno de los m�todos de la clase se comporta como se espera. Es decir, en funci�n de alg�n valor de entrada se eval�a el valor de retorno esperado; si la clase cumple con la especificaci�n, entonces JUnit devolver� que el m�todo de la clase pas� exitosamente la prueba; en caso de que el valor esperado sea diferente al que regres� el m�todo durante la ejecuci�n, JUnit devolver� un fallo en el m�todo correspondiente.