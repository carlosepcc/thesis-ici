\section{Pruebas}
Ya dise�ada e implementada la propuesta de soluci�n se debe proceder a su validaci�n. El presente cap�tulo precisa cada uno de los pasos que se tomaron para el aseguramiento de la calidad del sistema, iniciado desde la planificaci�n del proceso de software. Se ofrece un an�lisis de cada uno de los niveles de pruebas ejecutados, los t�rminos en que se realizaron las pruebas y los resultados alcanzados

\subsection{Estrategia de pruebas}
Una estrategia de prueba de software proporciona una gu�a que describe los pasos que deben realizarse como parte de la prueba. Debe ser lo suficientemente flexible para promover un uso personalizado de la prueba \citep{pressman_isw_7ma}. Al probar el software se verifican los resultados de la prueba que se opera para buscar errores, anomal�as o informaci�n de atributos no funcionales del programa \citep{sommerville2011software}.\\
El proceso de prueba tiene dos metas distintas:
\begin{itemize}
  \item Demostrar al desarrollador y al cliente que el software cumple con los requerimientos.
  \item Encontrar situaciones donde el comportamiento del software sea incorrecto, indeseable o no est� de acuerdo con su especificaci�n.
\end{itemize}
Una estrategia de prueba est� compuesta por niveles, tipos, m�todos y t�cnicas de pruebas, as� como los casos de prueba \citep{sommerville2011software}.

La metodolog�a XP propone un modelo en el que lo primero que se escribe son las pruebas que el sistema debe pasar. Luego, el desarrollo debe ser el m�nimo necesario para pasar las pruebas previamente definidas. Las pruebas a las que se refiere esta pr�ctica son las pruebas unitarias realizadas por los desarrolladores. La definici�n condiciona el desarrollo del sistema. XP divide las pruebas de software o de sistema en dos grupos, las pruebas unitarias y las pruebas de aceptaci�n \citep{sommerville2011software}.

Uno de los pilares de \ac{xp} es el proceso de pruebas. Esta metodolog�a de desarrollo anima a probar constantemente tanto como sea posible. Esto permite aumentar la calidad de los sistemas reduciendo el n�mero de errores no detectados y disminuyendo el tiempo transcurrido entre la aparici�n de un error y su detecci�n. Tambi�n permite aumentar la seguridad de evitar efectos colaterales no deseados a la hora de realizar  dise�ada por los programadores, y pruebas de aceptaci�n o pruebas funcionales destinadas a evaluar si al final de una iteraci�n se consigui� la funcionalidad requerida dise�adas por el cliente final \citep{gutierrez2006pruebas}.
Con el objetivo de comprobar que las funcionalidades implementadas funcionan de acuerdo a las especificaciones descritas por el cliente, se realizaron pruebas unitarias. %TODO y pruebas de sitema

\subsection{Pruebas unitarias}
Las pruebas unitarias o unittesting son una forma de comprobar que un fragmento de c�digo funciona correctamente.Son peque�os tests que validan el comportamiento de un objeto y la l�gica \citep{gutierrez2006pruebas}. \ac{xp} plantea la realizaci�n de pruebas unitarias continuas, frecuentemente repetidas y automatizadas, incluyendo pruebas de regresi�n, y aconseja escribir el c�digo de la prueba antes de la codificaci�n \citep{kniberg2007scrumyxp}.\\
Para la aplicaci�n de las pruebas unitarias a la soluci�n se emple� el framework JUnit el cual permite realizar la ejecuci�n de clases Java de manera controlada, para poder evaluar si el funcionamiento de cada uno de los m�todos de la clase se comporta como se espera. Es decir, en funci�n de alg�n valor de entrada se eval�a el valor de retorno esperado; si la clase cumple con la especificaci�n, entonces JUnit devolver� que el m�todo de la clase pas� exitosamente la prueba; en caso de que el valor esperado sea diferente al que regres� el m�todo durante la ejecuci�n, JUnit devolver� un fallo en el m�todo correspondiente.

% \clearpage

\subsection{Pruebas de sistema}
Esta prueba tiene como objetivo verificar que se han integrado adecuadamente todos los elementos del sistema y que realizan las operaciones apropiadas funcionando como un todo \citep{sommerville2011software}.
\subsubsection*{M�todo de prueba de caja negra:}
Las pruebas de caja negra se centran en los requisitos funcionales del sistema. Con estas pruebas se intentan encontrar funciones incorrectas o ausentes, errores de interfaz, errores en estructuras de datos o en acceso a base de datos externas y errores de rendimiento. Se centran en qu� hace el software y no en c�mo lo hace \citep{pressman_isw_7ma}.
Seg�n \citep{pressman_isw_7ma} existen varias t�cnicas para realizar este tipo de pruebas,
se seleccion� la t�cnica partici�n equivalente, la cual permite comprobar los valores v�lidos e
inv�lidos de las entradas existentes en la aplicaci�n.
\paragraph*{Partici�n equivalente:} m�todo que divide el campo de entrada de un programa en clases
de datos de los que se pueden derivar casos de prueba. Se dirige a la definici�n de
casos de prueba que descubran clases de errores, reduciendo as� el n�mero total de
casos de prueba que hay que desarrollar.
La figura \ref{fig:tests-system-coverage} a continuaci�n muestra el resultado de la aplicaci�n de esta prueba.

\clearpage
\begin{figure}[htp]
	\centering
	\includegraphics[width=0.8\textwidth]{images/test/no-conf.png}
	\caption{No conformidades detectadas}
	\label{fig:tests-system-coverage}
\end{figure}

% \subsection{Pruebas de aceptaci�n}
Las pruebas de aceptaci�n son definidas por el usuario del sistema y preparadas por el equipo de desarrollo, aunque la ejecuci�n y aprobaci�n final corresponden al usuario. Estas pruebas van dirigidas a comprobar que el sistema cumple los requisitos de funcionamiento esperado, recogidos en el cat�logo de requisitos y en los criterios de aceptaci�n del sistema de informaci�n, y conseguir as�? la aceptaci�n final del sistema por parte del usuario \citep{gutierrez2006pruebas}.
A continuaci�n se muestran algunas de las pruebas de aceptaci�n realizadas.

\begin{acceptancetest}[t:hu1p1] % label in brackets
   \testcasecode{HU1\_P1}
   \testcasedescription{Prueba para la funcionalidad: Autenticar usuario. Prueba que s�lo se puede acceder a las funcionalidades del sistema si se ha realizado una autenticaci�n exitosa primero}
   \testcaseexeccond{El usuario no est� autenticado en el sistema.}
   \testcaseexecstep{
      \begin{enumerate}
         \item Se navega a la p�gina donde se encuentra el formulario de autenticaci�n.
         \item Se ingresan los credenciales correctos en el formulario de autenticaci�n.
         \item Se inicia la autenticaci�n a trav�z del bot�n de acci�n del formulario o la tecla Enter.
         \item Se recarga la p�gina.
      \end{enumerate}
   }
   \testcaseexpresult{El usuario queda autenticado en el sistema. Se muestran los elementos de la interfaz que permiten acceder a las funcionalidades del sistema y la informaci�n del usuario autenticado justo despu�s de culminar el proceso de autenticaci�n. Evaluaci�n de la prueba: Satisfactoria}
   \testcasename{Iniciar sesi�n con credenciales correctos}
   \testcaseuserstory{1}
\end{acceptancetest}

\begin{acceptancetest}[t:hu1p2] % label in brackets
   \testcasecode{HU1\_P2}
   \testcasedescription{Prueba para la funcionalidad: Autenticar usuario. Prueba que s�lo se puede acceder a las funcionalidades del sistema si se ha realizado una autenticaci�n exitosa primero}
   \testcaseexeccond{El usuario no est� autenticado en el sistema.}
   \testcaseexecstep{
      \begin{enumerate}
         \item Se navega a la p�gina de inicio de sesi�n.
         \item Se ingresan credenciales incorrectos en el formulario de autenticaci�n.
         \item Se inicia la autenticaci�n a trav�z del bot�n de acci�n del formulario o la tecla Enter.
      \end{enumerate}
   }
   \testcaseexpresult{El usuario no queda autenticado en el sistema. Se muestra un aviso indicando que los credenciales son incorrectos. Evaluaci�n de la prueba: Satisfactoria}
   \testcasename{Iniciar sesi�n con credenciales incorrectos}
   \testcaseuserstory{1}
\end{acceptancetest}

% 
\begin{acceptancetest}[t:hu1p3] % label in brackets
   \testcasecode{HU1\_P3}
   \testcasename{Cerrar sesi�n}
   \testcaseuserstory{1}
   \testcasedescription{Prueba para la funcionalidad: Autenticar usuario. Prueba que el usuario puede cerrar una sesi�n iniciada.}
   \testcaseexeccond{El usuario est� autenticado en el sistema.}
   \testcaseexecstep{
      \begin{enumerate}
         \item Se inicia el cierre de sesi�n desde el men� de usuario en la aplicaci�n cliente.
         \item Se confirma la acci�n en un cuadro de di�logo.
         \item Se recarga la p�gina.
      \end{enumerate}
   }
   \testcaseexpresult{El usuario deja de estar autenticado en el sistema.
      Evaluaci�n de la prueba: Satisfactoria}
\end{acceptancetest}

\begin{acceptancetest}[t:hu4p1]
   \testcasecode{HU4\_P1}
   \testcaseuserstory{2}
   \testcasename{Listar denuncias}
   \testcasedescription{Prueba para la funcionalidad: Listar denuncias. Prueba que el usuario puede obtener una lista de denuncias sobre las que posee permiso de lectura solamente.}
   \testcaseexeccond{El usuario est� autenticado en el sistema.}
   \testcaseexecstep{
      \begin{enumerate}
         \item Se navega hacia la p�gina para gesti�n de denuncias.
         \item Se presiona el bot�n de acci�n "Refrescar".
      \end{enumerate}
   }
   \testcaseexpresult{Se muestra una tabla con filas llenas con los datos de las denuncias a las que el usuario autenticado tiene acceso,  o vac�a si no existe ninguna tupla en la base de datos que satisfaga esa condici�n, en cuyo caso se informa al usuario que no hay datos disponibles.
      Evaluaci�n de la prueba: Satisfactoria}
\end{acceptancetest}

% LISTAR
\begin{acceptancetest}[t:hu2p1]
   \testcasecode{HU2\_P1}
   \testcaseuserstory{2}
   \testcasename{Listar usuarios}
   \testcasedescription{Prueba para la funcionalidad: Listar usuarios. Prueba que el usuario puede obtener una lista de usuarios sobre los que posee permiso de lectura solamente.}
   \testcaseexeccond{El usuario est� autenticado en el sistema.}
   \testcaseexecstep{
      \begin{enumerate}
         \item Se navega hacia la p�gina para gesti�n de usuarios.
         \item Se presiona el bot�n de acci�n "Refrescar".
      \end{enumerate}
   }
   \testcaseexpresult{Se muestra una tabla con filas llenas con los datos de los usuarios a los que el usuario autenticado tiene acceso de lectura,  o vac�a si no existe ninguna tupla en la base de datos que satisfaga esa condici�n, en cuyo caso se informa al usuario que no hay datos disponibles.
      Evaluaci�n de la prueba: Satisfactoria}
\end{acceptancetest}

% LISTAR DECLARACIONES
\begin{acceptancetest}[t:hu8p1]
   \testcasecode{HU8\_P1}
   \testcaseuserstory{2}
   \testcasename{Listar declaraciones}
   \testcasedescription{Prueba para la funcionalidad: Listar declaraciones. Prueba que el usuario puede obtener una lista de declaraciones sobre las que posee permiso de lectura solamente.}
   \testcaseexeccond{El usuario est� autenticado en el sistema.}
   \testcaseexecstep{
      \begin{enumerate}
         \item Se navega hacia la p�gina para gesti�n de declaraciones.
         \item Se presiona el bot�n de acci�n "Refrescar".
      \end{enumerate}
   }
   \testcaseexpresult{Se muestra una tabla con filas llenas con los datos de las declaraciones a las que el usuario autenticado tiene acceso de lectura,  o vac�a si no existe ninguna tupla en la base de datos que satisfaga esa condici�n, en cuyo caso se informa al usuario que no hay datos disponibles.
      Evaluaci�n de la prueba: Satisfactoria}
\end{acceptancetest}

% LISTAR RESOLUCIONES
\begin{acceptancetest}[t:hu5p1]
   \testcasecode{HU5\_P1}
   \testcaseuserstory{2}
   \testcasename{Listar resoluciones}
   \testcasedescription{Prueba para la funcionalidad: Listar resoluciones. Prueba que el usuario puede obtener una lista de resoluciones sobre las que posee permiso de lectura solamente.}
   \testcaseexeccond{El usuario est� autenticado en el sistema.}
   \testcaseexecstep{
      \begin{enumerate}
         \item Se navega hacia la p�gina para gesti�n de resoluciones.
         \item Se presiona el bot�n de acci�n "Refrescar".
      \end{enumerate}
   }
   \testcaseexpresult{Se muestra una tabla con filas llenas con los datos de las resoluciones a las que el usuario autenticado tiene acceso de lectura,  o vac�a si no existe ninguna tupla en la base de datos que satisfaga esa condici�n, en cuyo caso se informa al usuario que no hay datos disponibles.
      Evaluaci�n de la prueba: Satisfactoria}
\end{acceptancetest}

\begin{figure}[htp]
   \centering
   \includegraphics[width=1\textwidth]{images/test/acceptance-tests-coverage.png}
   \caption{Resumen de las pruebas ed aceptaci�n.}
   \label{fig:acceptancetest}
\end{figure}