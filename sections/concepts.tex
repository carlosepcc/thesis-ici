\section[Conceptos]{Principales conceptos asociados al proceso de Comisi�n Disciplinaria en la Facultad 4}\label{concepts}
\comment[carlosepc]{Buscar y citar las fuentes faltantes de las explicaciones de los conceptos.}
\paragraph{Denuncia:} Es la acci�n y efecto de denunciar (avisar, noticiar, declarar la irregularidad o ilegalidad de algo, delatar). La denuncia puede realizarse ante las autoridades correspondientes (lo que implica la puesta en marcha de un mecanismo judicial) o de forma p�blica (s�lo con valor testimonial) \citep{rae}.
\paragraph{Comisi�n disciplinaria:} Conjunto de personas encargadas por una autoridad de velar por la buena conducta y disciplina \citep{rae}. En el contexto del proceso analizado en este trabajo se refiere al equipo conformado por un jefe y un secretario, ambos profesores, que se encarga de la resoluci�n de un caso disciplinario. Si la indisciplina asociada al caso fue realizada en la residencia, pasan a formar parte de la comisi�n disciplinaria el representante de la residencia, quien es un trabajador de la misma, y un representante del edificio donde vive el estudiante.

\paragraph{Caso disciplinario:} Se crea cuando se aprueba una denuncia y se asigna una comisi�n disciplinaria para su an�lisis.

\paragraph{Gesti�n:} Del lat�n gest\u{i}o, el concepto de gesti�n hace referencia a la acci�n y a la consecuencia de administrar o gestionar algo. Al respecto, hay que decir que gestionar es llevar a cabo diligencias que hacen posible la realizaci�n de una operaci�n comercial o de un anhelo cualquiera \citep{definicionde}.

\paragraph{Sistema de gesti�n:}

\paragraph{Servidor web:} Un servidor web o Servidor HTTP es una pieza de software de comunicaciones que intermedia entre el servidor en el que est�n alojados los datos solicitados y el computador del cliente, permitiendo conexiones bidireccionales o unidireccionales, s�ncronas o as�ncronas, con cualquier aplicaci�n del cliente, incluso con los navegadores que traducen un c�digo (renderizable) a una p�gina web determinada. O sea, se trata de programas que median entre el usuario de Internet y el servidor en donde est� la informaci�n que solicita.

\paragraph{Aplicaci�n cliente:} Una aplicaci�n cliente es un paquete de software que funciona sobre el propio sistema operativo de un dispositivo (que puede ser por ejemplo un smartphone, una laptop o un equipo de escritorio). Es decir, que la aplicaci�n se instala y corre �dentro� del computador, o lo que solemos llamar tambi�n �de forma local�. En un computador, estas aplicaciones se instalan en el disco duro, donde guardan toda la informaci�n. 

\paragraph{Aplicaci�n web:}