\section{Tareas}
La metodolog�a de software XP plantea que la implementaci�n de un software se hace iterativamente.
Durante cada iteraci�n se desarrollan un conjunto de \ac{hu} definidas por el cliente y descritas por el equipo de desarrollo. En esta fase de implementaci�n las \ac{hu} se dividen en tareas, las cuales son asignadas a los programadores para ser implementadas durante la iteraci�n correspondiente \citep{Joskowicz2008}.\\
A continuaci�n, se muestran algunas de las tareas de ingenier�a a realizar:

\subsection{Iteraci�n I}
Para su desarrollo durante la primera iteraci�n se acuerda la selecci�n de las \ac{hu} con mayor prioridad,respetando la opini�n del cliente, cuya suma del tiempo total estimado de desarrollo no exceda el tiempo acordado para las iteraciones ( 2 semanas ). A continuaci�n los nombres de algunas de las \ac{hu} seleccionadas para la iteraci�n junto a las tareas que, una vez implementadas satisfactoriamente, dan paso a la realizaci�n de las pruebas de aceptaci�n:

\begin{itemize}
\item \textbf{HU4:} Realizar denuncia
\end{itemize}
% \begin{tasks}

% 	\item Implementar: Crear denuncia.
% 	\item Implementar: Modificar denuncia.
% 	\item Implementar: Eliminar denuncia.
% 	\item Implementar: Consultar denuncia.
% 	\item Implementar: Listar denuncias.
% 	\item Implementar: Buscar denuncia.
% 	\item Implementar: Exportar denuncia.
% 	\item Implementar: Autenticar usuario.
% 	\item Implementar: Asignar rol a usuario.
% 	\item Implementar: Crear rol
% 	\item Implementar: Modificar rol.
% 	\item Implementar: Eliminar rol.
% 	\item Implementar: Listar roles.
% 	\item Implementar: Crear resoluci�n decanal
% 	\item Implementar: Listar resoluciones decanales.
% 	\item Implementar: Consultar resoluci�n decanal.
% 	\item Implementar: Exportar resoluci�n decanal.
	
% \end{tasks}


\subsection{Iteraci�n II}
Para su desarrollo durante la segunda iteraci�n se acuerda la selecci�n de las \ac{hu} con prioridad de segundo orden,respetando la opini�n del cliente, cuyo cantidad no exceda la el n�mero de \ac{hu} que se pudieron desarrollar y probar exitosamente en la iteraci�n anterior, lo que se conoce como velocidad de desarrollo. Luego se verifica que las cantidad de tareas total asociadas a las \ac{hu} no excede tampoco la velocidad de desarrollo de la iteraci�n anterior. Y de esta manera se procede con todas las iteraciones siguientes. A continuaci�n los nombres de algunas de las \ac{hu} seleccionadas para la iteraci�n junto a las tareas que, una vez implementadas satisfactoriamente, dan paso a la realizaci�n de las pruebas de aceptaci�n:

\begin{itemize}
	
\item \textbf{HU1:} Autenticar usuario
\item \textbf{HU2:} Listar usuarios
\item \textbf{HU3:} Asignar rol a usuario
\end{itemize}

\begin{engineeringtask}[t:engtask1] % label in brackets
    \engtaskuserstory{1}
    \engtaskname{Autenticar usuario}
    \engtasktype{Desarrollo}
    \engtaskpointestimation{1} % d�as ideales de desarrollo
    \engtaskstartdate{16}{9}{2022} % day, month, year
    \engtaskenddate{17}{9}{2022}
    \engtaskprogrammer{\authorA}
    \engtaskdescription{
      Implementar la autenticaci�n basada en JSON Web Tokens (JWT). Deben enviarse en el token los datos del usuario autenticado:
    Nombre de usuario: String
     Nombre: String
     Rol: String
   % Permisos: String[] <[ROLE_C_DENUNCIA, ROLE_R_DENUNCIA, *]>
    Cargo: String <Profesor | Decano | Administrador | Estudiante | Trabajador | Secretario de Comunicaci�n de la FEU de la Facultad 4 | *>
   %  Sexo: String <M | F>
   %  Categor�a cient�fica: String <Dr.C | M.C | Ing | Lic>
    }
    
 \end{engineeringtask}

 \begin{engineeringtask}[t:engtask1] % label in brackets
    \engtaskuserstory{1}
    \engtaskname{Manejo de sesi�n}
    \engtasktype{Desarrollo}
    \engtaskpointestimation{1} % d�as ideales de desarrollo
    \engtaskstartdate{17}{9}{2022} % day, month, year
    \engtaskenddate{18}{9}{2022}
    \engtaskdescription{Implementar la persistencia y cierre de sesi�n usando una API web como Local Storage o Cookies para almacenar y recuperar los datos de una sesi�n iniciada, as� como para eliminarlos cuando el usuario decide cerrar la sesi�n.}
    \engtaskprogrammer{\authorA}
 \end{engineeringtask}
 % \begin{developmenttask}[t:devtask1] % label in brackets
 %     \devtaskuserstory{4}
 %     \devtaskname{Implementar: Autenticar usuario}
 %     \devtasktype{Desarrollo}
 %     \devtaskpointestimation{1}
 %     \devtaskstartdate{1}{10}{2022} % day, month, year
 %     \devtaskenddate{2}{10}{2022}
 %     \devtaskdescription{La autenticaci�n debe funcionar usando el servicio de autenticaci�n de la universidad.}
 %     \devtaskprogrammer{\authorA}
 %  \end{developmenttask}


% \begin{developmenttask}[t:devtask1] % label in brackets
%     \devtaskuserstory{4}
%     \devtaskname{Implementar: Autenticar usuario}
%     \devtasktype{Desarrollo}
%     \devtaskpointestimation{1}
%     \devtaskstartdate{1}{10}{2022} % day, month, year
%     \devtaskenddate{2}{10}{2022}
%     \devtaskdescription{La autenticaci�n debe funcionar usando el servicio de autenticaci�n de la universidad.}
%     \devtaskprogrammer{\authorA}
%  \end{developmenttask}