 \section{Introducci�n}
 
 El presente cap�tulo abordan las particularidades del sistema de gesti�n a desarrollar. Para registrar las principales caracter�sticas se hace uso de los \ac{RF} y \ac{RnF}. Estos describen las funcionalidades y atributos de calidad que debe poseer el software. 
 	
 En el cap�tulo \ref{chap:chapter1}, se seleccion� la metodolog�a XP como gu�a para el desarrollo del software; por lo tanto, se utilizan las \ac{HU} como herramienta para una descripci�n detallada de los \ac{RF} y la confecci�n del plan de iteraciones. Mediante el uso de este �ltimo, se proceder� a la estimaci�n del tiempo requerido para la culminaci�n del desarrollo del sistema y, con el uso de patrones de dise�o, se facilitar� la posterior descripci�n de las tarjetas \ac{CRC}.
 
 
 
 \section{Requisitos funcionales}
 En ingenier�a de software, los \ac{RF} definen un sistema o sus componentes. Los \ac{RF} describen la funci�n que un software debe realizar, ya sean c�lculos, manipulaci�n de datos, procesos de negocios, entre otros.
 
 Los \ac{RF} ayudan a capturar los comportamientos planificados para un sistema, este comportamiento puede ser expresado como una funci�n, servicio o tarea que un software debe realizar~\citep{requisitos}. A continuaci�n, se exponen los diferentes \ac{RF} planteados por el usuario:
