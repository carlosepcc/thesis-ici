\section{Tareas de Ingenier�a}
La metodolog�a de software XP plantea que la implementaci�n de un software se hace iterativamente.
Durante cada iteraci�n se desarrollan un conjunto de HU definidas por el cliente y descritas por el equipo de desarrollo. En esta fase de implementaci�n las HU se dividen en tareas de ingenier�a, las cuales son asignadas a los programadores para ser implementadas durante la iteraci�n correspondiente \citep{Joskowicz2008}.\\
A continuaci�n, se muestran algunas de las tareas de ingenier�a:


\subsection{Iteraci�n I}

Dentro de la primera iteraci�n se encuentran las funcionalidades de ``Crear denuncia'' y  ``Modificar denuncia'', correspondientes a las historias de usuario 1 y 2. Las tareas definidas para la iteraci�n son:

\begin{itemize}
	\item Tarea N.1: Cargar.
\end{itemize}

\begin{engineeringtask}
	\engtaskuserstory {1}
	\engtaskname {Crear denuncia}
	\engtasktype {Desarrollo}
	\engtaskpointestimation {0.2}
	\engtaskstartdate {2}{9}{2022} % day , month , year
	\engtaskenddate {5}{9}{2022}
	\engtaskdescription {Crear m�todo que sea capaz de crear una denuncia}
	\engtaskprogrammer {\authorF}
\end{engineeringtask}