\section{Requisitos}\label{s:req}

\subsection{Requisitos funcionales}
En ingenier�a de software, los \ac{rf} definen un sistema o sus componentes; describen la funci�n que un software debe realizar, ya sean c�lculos, manipulaci�n de datos, procesos de negocios, entre otros.
Ayudan adem�s a capturar los comportamientos planificados para un sistema, este comportamiento puede ser expresado como una funci�n, servicio o tarea que un software debe realizar \citep{acanda2018cdis}. A continuaci�n, se exponen los diferentes \ac{rf} planteados por el usuario:
\begin{RF}
\item Autenticar Usuario
\item Asignar rol a usuario
\\
\item Listar denuncias %w
\item Crear denuncia
\item Modificar denuncia
\item Eliminar denuncia
\item Buscar denuncia
\item Exportar tabla de denuncias
%\item Imprimir denuncia
%\item Imprimir listado de denuncias
\\
\item Listar resoluciones %w
\item Crear resoluci�n %w
%\item Modificar resoluci�n 
\item Eliminar resoluci�n %w
\item Exportar resoluci�n
\\
\item Listar comisiones %w
\item Crear comisi�n %w
%\item Modificar comisi�n
\item Eliminar comisi�n %w
\item Buscar comisi�n %w
\\
\item Listar declaraciones
\item Crear declaraci�n
\item Modificar declaraci�n
\item Eliminar declaraci�n
\item Buscar declaraci�n
%\item Exportar tabla de declaraciones
\\
\item Listar casos disciplinarios
\item Crear caso disciplinario
\item Modificar caso disciplinario
\item Buscar caso disciplinario
\\
\item Exportar resoluci�n de caso
\item Exportar conclusi�n de la comisi�n
\\
\item Listar roles
\item Crear rol
\item Modificar rol
\item Eliminar rol
\item Buscar rol
%\item Exportar tabla de usuarios
\end{RF}

\subsection{Requisitos no funcionales}
Los \ac{rnf}, como su nombre sugiere, son aquellos requerimientos que no se refieren directamente a las funciones espec�ficas que proporciona el sistema, sino a las propiedades emergentes de este como la fiabilidad, el tiempo de respuesta y la capacidad de almacenamiento. De forma alternativa definen las restricciones del sistema \citep{sommerville2011software}.\\
Los requisitos no funcionales de la aplicaci�n a desarrollar son:
\\
\noindent Usabilidad:
\begin{RNF}
\item La \ac{gui} de la aplicaci�n cliente debe ser intuitiva y f�cil de usar por cualquier persona con edad laboral y con conocimientos b�sicos de trabajo con software, de acuerdo a est�ndares modernos de dise�o de \ac{gui}.
\end{RNF}
Hardware y software:
\begin{RNF}[resume]
\item El software debe poder funcionar en un servidor con 4GB de \ac{ram} y 10GB de almacenamiento libre.
\item El cliente del sistema debe ser compatible con las �ltimas versiones estables de los principales navegadores web: Google Chrome, navegadores basados en \gls{chromium}, Mozilla Firefox y Safari.
%TODO Seguridad?  \item El usuario debe poder acceder a las funcionalidades del sistema s�lo despu�s de estar autenticado.
% \item El usuario debe poder acceder solamente a las funcionalidades permitidas de acuerdo a su rol en el sistema.
\end{RNF}
Dise�o e implementaci�n:
\begin{RNF}[resume]
\item El sistema debe ser accesible a trav�s de una aplicaci�n web cliente.
\item Se deben usar lenguajes de programaci�n y herramientas com amplio soporte y estabilidad.
\end{RNF}
