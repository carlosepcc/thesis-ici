\section{Requisitos}\label{s:req}

\subsection{Requisitos funcionales}
En ingenier�a de software, los \ac{rf} definen un sistema o sus componentes; describen la funci�n que un software debe realizar, ya sean c�lculos, manipulaci�n de datos, procesos de negocios, entre otros.
Ayudan adem�s a capturar los comportamientos planificados para un sistema, este comportamiento puede ser expresado como una funci�n, servicio o tarea que un software debe realizar \citep{pressman2005software}. A continuaci�n, se exponen los diferentes \ac{rf} planteados por el usuario:
\begin{RF}
    \item Autenticar Usuario (Prioridad Alta)
    \item Asignar rol a usuario (Prioridad Baja)

    \item Listar denuncias (Prioridad Alta)
    \item Crear una denuncia (Prioridad Alta)
    \item Modificar una denuncia (Prioridad Alta)
    \item Eliminar denuncias (Prioridad Alta)
    \item Buscar denuncias (Prioridad Baja)
    \item Archivar denuncias (Prioridad Alta)

    \item Listar resoluciones decanales (Prioridad Alta)
    \item Crear una resoluci�n decanal (Prioridad Alta)
    \item Modificar una resoluci�n decanal (Prioridad Alta)
    %\item Eliminar resoluciones decanales (Prioridad Alta)

    \item Listar comisiones (Prioridad Alta)
    \item Crear comisiones (Prioridad Alta)
    \item Modificar una comisi�n (Prioridad Alta)
    \item Eliminar comisiones (Prioridad Alta)
    \item Buscar comisiones (Prioridad Baja)

    \item Listar declaraciones (Prioridad Alta)
    \item Crear declaraci�n (Prioridad Alta)
    \item Modificar declaraci�n (Prioridad Alta)
    \item Eliminar declaraciones (Prioridad Alta)
    \item Buscar declaraci�n (Prioridad Baja)

    \item Crear caso disciplinario (Prioridad Alta)
    \item Listar casos disciplinarios (Prioridad Alta)
    \item Modificar caso disciplinario (Prioridad lta)
    \item Buscar caso disciplinario (Prioridad Baja)
    \item Cerrar caso disciplinario (Prioridad Media)

    \item Listar roles (Prioridad Baja)
    \item Crear roles (Prioridad Baja)
    \item Modificar roles (Prioridad Baja)
    \item Eliminar roles (Prioridad Baja)
    \item Buscar roles (Prioridad Baja)

    \item Exportar resoluciones decanales (Prioridad Baja)
    \item Exportar denuncias (Prioridad Baja)
    \item Exportar declaraciones (Prioridad Baja)
    \item Exportar conclusiones de casos (Prioridad Baja)
    \item Exportar resoluciones de casos (Prioridad Baja)
\end{RF}

\subsection{Requisitos no funcionales}
Los \ac{rnf}, como su nombre sugiere, son aquellos requerimientos que no se refieren directamente a las funciones espec�ficas que proporciona el sistema, sino a las propiedades emergentes de este como la fiabilidad, el tiempo de respuesta y la capacidad de almacenamiento. De forma alternativa definen las restricciones del sistema \citep{sommerville2011software}.\\
Los requisitos no funcionales de la aplicaci�n a desarrollar son:
\\
\noindent \textbf{Usabilidad:}
\begin{RNF}
    \item La \ac{gui} de la aplicaci�n cliente debe ser intuitiva y f�cil de usar por cualquier persona con edad laboral y con conocimientos b�sicos de trabajo con software, de acuerdo a est�ndares modernos de dise�o de \ac{gui}.
\end{RNF}
\textbf{Hardware y software:}
\begin{RNF}[resume]
    \item El software debe poder funcionar en un servidor con 4GB de \ac{ram} y 10GB de almacenamiento libre.
    \item El cliente del sistema debe ser compatible con las �ltimas versiones estables de los principales navegadores web: Google Chrome, navegadores basados en \gls{chromium}, Mozilla Firefox y Safari.
    %TODO Seguridad?  \item El usuario debe poder acceder a las funcionalidades del sistema s�lo despu�s de estar autenticado.
    % \item El usuario debe poder acceder solamente a las funcionalidades permitidas de acuerdo a su rol en el sistema.
\end{RNF}
\textbf{Dise�o e implementaci�n:}
\begin{RNF}[resume]
    \item El sistema debe ser accesible a trav�s de una aplicaci�n web cliente.
    \item Se deben usar lenguajes de programaci�n y herramientas com amplio soporte y estabilidad.
\end{RNF}
