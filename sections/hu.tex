\section{Historias de usuario}
\label{s:hu}
Las HU ser�n representadas mediante tablas divididas por las siguientes secciones:
\begin{description}

\item N�mero: n�mero de la historia de usuario incremental en el tiempo;
\item Nombre de historia de usuario: el nombre de la historia de usuario ser�a para identificarlas mejor
entre los desarrolladores y el cliente;
\item[Usuario] El usuario que est� involucrado en el desarrollo de la HU;
\item[Iteraci�n asignada] N�mero de la iteraci�n;
\item[Prioridad en negocio] 
\begin{itemize}
\item Las historias de usuarios que son de funcionalidades imprescindibles en el desarrollo del sistema tienen prioridad alta;
\item Las historias de usuarios que son de funcionalidades que debe de tener el sistema, pero que
no son necesarias para su funcionamiento, tienen prioridad media;
\item Las historias de usuarios que son de funcionalidades auxiliares y que son independientes del
sistema, tienen prioridad baja.
\end{itemize}
\item Riesgo en desarrollo:
\begin{itemize}
\item Las historias de usuarios que, en caso de tener alg�n error de implementaci�n, puedan afectar
la disponibilidad del sistema, tienen riesgo de desarrollo alto;
\item Las historias de usuarios que puedan presentar errores y retrasan la entrega de la versi�n,
tienen riesgo de desarrollo medio;
\item Las historias de usuario que puedan presentar errores, pero estos son tratados con facilidad y
no afectan en desarrollo del proyecto, tienen riesgo de desarrollo bajo.
\end{itemize}
\item[Puntos estimados] tiempo estimado que se demorar� el desarrollo de la HU;
\item[Descripci�n] breve descripci�n de la HU;
\item[Observaciones] se�alamiento o advertencia del sistema;
\item[Prototipo de interfaz] prototipo de interfaz si aplica.
\end{description} \citep{Joskowicz2008}


Los t�tulos de las HU generadas son:
\begin{enumerate}[label=HU \arabic*:]
\item  Autenticar usuario.
\item  Gestionar usuario.
\item  Mostrar reportes al asesor.
\item  Gestionar denuncias.
\item  Mostrar clasificaci�n de las faltas.
\item  Gestionar comisiones disciplinarias.
\item  Gestionar expediente disciplinario.
\item  Gestionar dict�menes.
\item  Gestionar pr�rroga.
\item  Gestionar circunstancias modificativas.
\item  Exportar a PDF.
\end{enumerate}
A continuaci�n se presentan las historias de usuarios identificadas en la investigaci�n
\begin{userstory}
	\storyname{Crear denuncia}
	\storyuser{\userY}
	\storyiter{1}
	\storypriority{ High }
	\storyrisk{ Low }
	\storypoints{0.8}
	\storyprogrammer{\authorA }
	\storydescription{}
	\storyobservation{\lipsum[1]}
\end{userstory}
