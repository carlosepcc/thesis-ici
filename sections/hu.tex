\section{Historias de usuario}
\label{s:hu}
Las HU ser�n representadas mediante tablas divididas por las siguientes secciones:
\begin{description}

	\item [N�mero] Identificador entero incremental en el tiempo;
	\item [Nombre de historia de usuario] Identificador alfanum�rico para su uso
	      entre los desarrolladores y el cliente;
	\item[Usuario] Nombre y apellidos de la persona involucrada en el desarrollo de la \ac{hu};
	\item[Iteraci�n asignada] Identificador entero perteneciente a la iteraci�n en la cual se planea implementar la funcionalidad descrita en la \ac{hu};
	\item[Prioridad en negocio] Las historias de usuarios que describen funcionalidades imprescindibles en el desarrollo del sistema tienen prioridad alta; aquellas que debe tener el sistema, pero que no son necesarias para su funcionamiento, prioridad media; y auxiliares y que son independientes del sistema, prioridad baja.
	\item[Riesgo en desarrollo] Las historias de usuarios que, en caso de tener alg�n error de implementaci�n, puedan afectar la disponibilidad del sistema, tienen un riesgo de desarrollo alto; las \ac{hu} que puedan presentar errores y retrasan la entrega de la versi�n tienen riesgo de desarrollo medio; y las que puedan presentar errores, pero estos son tratados con facilidad y no afectan en desarrollo del proyecto, tienen riesgo de desarrollo bajo.

	\item[Puntos estimados] Tiempo estimado que tardar� el desarrollo de la \ac{hu};
	\item[Descripci�n] Breve descripci�n de \ac{hu};
	\item[Observaciones] Se�alamiento o advertencia del sistema;
	\item[Prototipo de interfaz] Prototipo de interfaz si aplica.
\end{description} \citep{Joskowicz2008}


Los t�tulos de las \ac{hu} generadas son:
\begin{enumerate}[label=HU \arabic*:]
	\item Autenticar Usuario
	\item Listar usuarios
	\item Asignar rol a Usuario
	      \\
	\item Gestionar denuncias
	      \\
	\item Gestionar resoluciones decanales %w
	\item Exportar resoluci�n decanal
	      \\
	\item Gestionar comisiones
	      \\
	\item Gestionar declaraciones
	%\item Exportar declaraci�n
	      \\
	\item Gestionar casos disciplinarios
	      \\
	\item Exportar conclusi�n de la comisi�n
	\item Exportar resoluci�n de caso
	      \\
	\item Gestionar roles
\end{enumerate}
A continuaci�n se presentan algunas de las historias de usuarios identificadas en la investigaci�n

\begin{userstory}
	\storyname{Asignar rol a usuario}
	\storyuser{Administrador}
	\storyiter{1}
	\storypriority{ Alta }
	\storyrisk{ Bajo }
	\storypoints{\hpa}
	\storyprogrammer{\authorA }
	\storydescription{Como administrador quiero asignar un rol a un usuario para que pueda realizar las acciones que le corresponden.}
	\storyobservation{}
	\storyinterface{
		% Formulario de asignaci�n de rol:\\
		% \includegraphics[scale=0.5]{images/prototypes/cdis-assignrole-capture.png}
	}
\end{userstory}
