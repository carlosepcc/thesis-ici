\section{Historias de usuario}
\label{s:hu}
Las HU ser�n representadas mediante tablas divididas por las siguientes secciones:
\begin{description}

\item [N�mero] Identificador entero incremental en el tiempo;
\item [Nombre de historia de usuario] Identificador alfanum�rico para su uso 
entre los desarrolladores y el cliente;
\item[Usuario] Nombre y apellidos de la persona involucrada en el desarrollo de la \ac{hu};
\item[Iteraci�n asignada] Identificador entero perteneciente a la iteraci�n en la cual se planea implementar la funcionalidad descrita en la \ac{hu};
\item[Prioridad en negocio] Las historias de usuarios que describen funcionalidades imprescindibles en el desarrollo del sistema tienen prioridad alta; aquellas que debe tener el sistema, pero que no son necesarias para su funcionamiento, prioridad media; y auxiliares y que son independientes del sistema, prioridad baja.
\item[Riesgo en desarrollo] Las historias de usuarios que, en caso de tener alg�n error de implementaci�n, puedan afectar la disponibilidad del sistema, tienen un riesgo de desarrollo alto; las \ac{hu} que puedan presentar errores y retrasan la entrega de la versi�n tienen riesgo de desarrollo medio; y las que puedan presentar errores, pero estos son tratados con facilidad y no afectan en desarrollo del proyecto, tienen riesgo de desarrollo bajo.

\item[Puntos estimados] Tiempo estimado que tardar� el desarrollo de la \ac{hu};
\item[Descripci�n] Breve descripci�n de \ac{hu};
\item[Observaciones] Se�alamiento o advertencia del sistema;
\item[Prototipo de interfaz] Prototipo de interfaz si aplica.
\end{description} \citep{Joskowicz2008}


Los t�tulos de las \ac{hu} generadas son:
\begin{enumerate}[label=HU \arabic*:]
\item Autenticar usuario.
\item Crear denuncia
\item Actualizar denuncia
\item Eliminar denuncia
\item Crear resoluci�n
\item Exportar resoluci�n.
\end{enumerate}
A continuaci�n se presentan las historias de usuarios identificadas en la investigaci�n
\begin{userstory}
	\storyname{Crear denuncia}
	\storyuser{\authorA}
	\storyiter{1}
	\storypriority{ Alta }
	\storyrisk{ Bajo }
	\storypoints{1}
	\storyprogrammer{\authorA }
	\storydescription{El sistema debe permitir registrar una denuncia en el sistema. Para ello se le debe brindar al usuario un formulario en el que ingresar los datos de la nueva denuncia y botones para registrarla en el sistema o cancelar la acci�n.
	Los datos a ingresar en el formulario deben ser:
		\begin{itemize}
		\item Estudiantes involucrados
		\item Fecha de ocurrencia de los hechos descritos en la denuncia
		\item Hora de ocurrencia de los hechos descritos en la denuncia
		\item Descripci�n
		\item Adjuntos (\textit{opcional})
		\end{itemize}}
	\storyobservation{Para registrar una denuncia tambi�n se necesita la informaci�n del denunciante y la fecha de la denuncia, pero se debe tomar el usuario autenticado como denunciante y la fecha de creaci�n de la denuncia en el sistema y no es necesario mostrar campos para esa informaci�n en el formulario }
\end{userstory}
\begin{userstory}
	\storyname{Crear comisi�n disciplinaria}
	\storyuser{\authorA}
	\storyiter{1}
	\storypriority{ Alta }
	\storyrisk{ Medio }
	\storypoints{1}
	\storyprogrammer{\authorA }
	\storydescription{El sistema debe permitir registrar una comisi�n disciplinaria. Para ello se le debe brindar al usuario un formulario en el que ingresar los datos de la nueva comisi�n disciplinaria y botones para registrarla en el sistema o cancelar la acci�n.
	Los datos a ingresar en el formulario deben ser:
		\begin{itemize}
		\item Presidente de comisi�n
		\item Secretario de comisi�n
		\end{itemize}}
	\storyobservation{Para registrar una comisi�n tambi�n se necesita la informaci�n de la resoluci�n a la que pertenece, pero, si se permite crear las comisiones en el formulario que se brinda para registrar o modificar una resoluci�n decanal, se debe tomar la resoluci�n a crear o editar como resoluci�n a la que pertenece la comisi�n creada}
\end{userstory}
\begin{userstory}
	\storyname{Crear resoluci�n}
	\storyuser{\authorA}
	\storyiter{1}
	\storypriority{ Alta }
	\storyrisk{ Bajo }
	\storypoints{0.8}
	\storyprogrammer{\authorA}
	\storydescription{El sistema debe permitir registrar una resoluci�n decanal en el sistema. Para ello se le debe brindar al usuario un formulario en el que ingresar los datos de la nueva resoluci�n decanal y botones para registrarla en el sistema o cancelar la acci�n.
	Los datos a ingresar en el formulario deben ser:
		\begin{itemize}
		\item Curso. Puede ser un n�mero correspondiente al a�o o una cCadena de caracteres que lo identifica.
		\item Comisiones disciplinarias
		\end{itemize}}
	\storyobservation{Para registrar una resoluci�n decanal tambi�n se necesita la informaci�n del decano que la emite, pero se debe tomar el usuario autenticado como decano que emite la resoluci�n decanal.}
\end{userstory}
