\section*[Introducci�n]{Introducci�n del cap�tulo}

En el presente cap�tulo se engloban aspectos relacionados con el objeto de estudio definido para el problema planteado. El an�lisis de algunas metodolog�as, procedimientos, herramientas existentes para el desarrollo de sistemas web y la observaci�n de aplicaciones hom�logas; permitir� la selecci�n de las tecnolog�as adecuadas para el desarrollo del \cdis\ y contar con un an�lisis de sistemas existentes que realizan funcionalidades similares.
Sobre la base de los elementos expuestos anteriormente se formula el siguiente problema de investigaci�n: �C�mo contribuir a la agilizaci�n del proceso de comisi�n disciplinaria en la facultad 4? Para la realizaci�n de la investigaci�n se define como objeto de estudio: el proceso de comisi�n disciplinaria qe se lleva a cabo en la facultad 4.
Para dar soluci�n al problema planteado, se define como objetivo general: desarrollar un sistema para la gesti�n del proceso de comisi�n disciplinaria en la Facultad 4 \ac{uci}.

Para dar cumplimiento al objetivo general antes mencionado, se dar� cumplimiento a los siguientes objetivos espec�ficos:

\begin{enumerate}
	\item Describir el estado actual de las herramientas dirigidas a la gesti�n de procesos disciplinarios en casas de altos estudios.
	\item Definir las tecnolog�as, herramientas y metodolog�a a utilizar en la implementaci�n de un sistema de gesti�n para el proceso de comisi�n disciplinaria en la facultad 4.	\item Dise�ar las funcionalidades sistema de gesti�n para el proceso de comisi�n disciplinaria en la facultad 4.
	\item Implementar y validar las funcionalidades del sistema de gesti�n para el proceso de comisi�n disciplinaria en la facultad 4
\end{enumerate}

\paragraph{Hip�tesis}
Con el desarrollo de un sistema de gesti�n para el proceso de comisi�n disciplinaria en la facultad 4 se contribuir� a la mejora del proceso de apoyo a la toma de decisiones.\\
Se define como \emph{variable independiente}: m�dulo de procesamiento estad�stico de informaci�n y como  \emph{variable dependiente}: proceso de apoyo a la toma de decisiones.
