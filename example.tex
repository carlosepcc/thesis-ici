\documentclass[spanish,noloa,nolol,pagelimitmode=flex,authorshippage=ltd]{thesis} [2015/07/11 v1.0.1]

	% custom commands %
\newcommand{\uci}{Universidad de las Ciencias Inform�ticas }
\newcommand{\fac}{Facultad 4 }
\newcommand{\cdis}{Sistema de Gesti�n para el Proceso de Comisi�n Disciplinaria en la \fac de la \uci }
\newcommand{\cdisen}{Management System for the Disciplinary Commission Process at Faculty 4 of the University of Computer Sciences }
\newcommand{\stripe}{\rowcolor[HTML]{EDEBF1}}
\newcommand{\theader}{\rowcolor[HTML]{CBCEFB}}
\newcommand{\authorC}{Carlos Enrique P�eiro C�rdenas}
\newcommand{\authorF}{�ngel Luis Fumero S�nchez}
\newcommand{\tutorY}{Yordankis Matos L�pez}
% set up lists
\newlist{RF}{enumerate}{1}
\setlist[RF]{label=RF \arabic*:}
\newlist{RNF}{enumerate}{1}
\setlist[RNF]{label=RNF \arabic*:}
	
	%	T�TULO
\title{\cdis}
\ucicenter{L�nea de Investigaci�n de Inform�tica Aplicada a la Sociedad} \facultynum{4}

	%	AUTOR�A Y TUTOR�A
\addauthor{\authorF}
\addauthor{\authorC}
\addtutor{MSc. \tutorY}

	%	PENSAMIENTO
\thought{"Lo que est� definido en el juicio, ser� de seguro bien puesto en los labios"\\
 Jos� Mart�}
 
	%	DEDICATORIA
\dedicatory{A quienes nos apoyaron.}

	%	AGRADECIMIENTOS
\acknowledgment{Les estamos sumamente agradecidos a nuestros familiares, amigos, profesores y a rtodas las personas e instituciones que nos sirvieron de apoyo, ayuda e inspiraci�n.}

\abstract{La Universidad de las Ciencias Inform�ticas, controla las indisciplinas rigi�ndose por la Resoluci�n
Ministerial No. 240 de 2007. Para lograr estos par�metros se auxilia de diferentes comisiones
disciplinarias que hacen constar en cada expediente las normativas vigentes, disponiendo de un grupo de
profesores y estudiantes integrados al trabajo disciplinario en sentido general.
Por consiguiente, la investigaci�n se propone como tema: \cdis, y tiene como objetivo: Desarrollar un sistema que
permita gestionar la informaci�n asociada al proceso de las comisiones disciplinarias de la \fac.
Los resultados que se avalan en la tesis se materializan a partir del desarrollo de un sistema que posibilita
gestionar, confeccionar y estandarizar las comisiones disciplinarias en la unidad docente de manera
organizada y controlada, eliminando los errores subjetivos que se presentan con frecuencia. Se utilizan en
la investigaci�n los m�todos te�ricos, emp�ricos y matem�tico estad�sticos que permitieron analizar el
problema propuesto, as� como las herramientas y artefactos que complementan la estructura y
organizaci�n de la propuesta investigativa.
\paragraph{Palabras clave:}\emph{comisi�n disciplinaria}, \emph{expediente disciplinario}, \emph{gesti�n}, \emph{sistema de gesti�n}
\section*{Abstract}
The University of Informatics Sciences, controls the indisciplines governed by the Ministerial Resolution
No. 240 of 2007. To achieve these parameters is assisted by different disciplinary committees that record
in each file the current regulations, having a group of teachers and students integrated to disciplinary work
in a general sense.
Therefore, the research is proposed as a topic: "\cdisen" , and aims to: Develop a system to manage the information associated with
the process of the disciplinary commissions of the Faculty 4.
The results that are endorsed in the thesis materialize from the development of a system that makes it
possible to manage, make and standardize the disciplinary commissions in the teaching unit in an
organized and controlled manner, eliminating subjective errors that occur frequently. Theoretical, empirical
and mathematical mathematical methods are used in the investigation that allowed to analyze the
proposed problem, as well as the tools and artifacts that complement the structure and organization of the
research proposal.
}
\keywords{disciplinary commission, disciplinary record, management, management system}

\newglossaryentry{casod}{name=caso disciplinario, description={Se crea cuando se aprueba una denuncia y se asigna una comis��n disciplinaria para su an�lisis.}}
\newglossaryentry{comisiond}{name=comisi�n disciplinaria, description={Equipo conformado por un jefe y un secretario,  que son profesores, que se encarga de la resoluci�n de un caso disciplinario. Si la indisciplina asociada al caso fue realizada en la residencia, pasan a formar parte de la comisi�n disciplinaria el representante de la residencia. que es un trabajador de la residencia, y un representante del edificio donde vive el estudiante.}}
\newglossaryentry{JSON}{name=JSON, description={Formato de texto sencillo para el intercambio de datos. Se trata de un subconjunto de la notaci�n literal de objetos de \ac{JS}.}}
\newacronym{uci}{UCI}{Universidad de las Ciencias Inform�ticas}
\newacronym{cdis}{CDIS}{Sistema de Gesti�n para el Proceso de Comisi�n Disciplinaria en la Facultad 4 de la Universidad de las Ciencias Inform�ticas}
\newacronym{xp-en}{XP}{Extreme Programming}
\newacronymeng{xp}{XP}{Programaci�n Extrema}
\newacronym{mes}{MES}{Ministerio de Educaci�n Superior}
\newacronym{rf}{RF}{requisitos funcionales}
\newacronym{rnf}{RnF}{requisitos no funcionales}
\newacronym{hu}{HU}{Historias de usuario}
\newacronym{cdis-2}{CDis}{Sistema inform�tico para la gesti�n de informaci�n de expedientes disciplinarios de la Facultad 2 en su versi�n 1.0}
\newacronym{codis}{CODIS}{Sistema para la informatizaci�n del proceso
de Comisi�n Disciplinaria de la Facultad 3}

\newacronym{crc}{CRC}{Clase-Responsabilidad-Colaboraci�n}
\newacronym{vp}{VP-UML}{Visual Paradigm}
\newacronymeng{npm}{NPM}{Gestor de paquetes de Node}

\newacronym{cpu-en}{CPU}{Central Processing Unit}
\newacronymeng{cpu}{CPU}{Unidad de Procesamiento Central}
\newacronymeng{case}{CASE}{Ingenier�a de Software Asistida por Computadora}

\newacronym{api-en}{API}{Application Programming Interface}
\newacronymeng{api}{API}{Interfaz de Programaci�n de Aplicaciones}


\newacronym{js}{JS}{JavaScript}

\newacronym{url-en}{URL}{Uniform Resource Locator}
\newacronymeng{url}{URL}{Localizador Uniforme de Recursos}	
\newacronym{rup-en}{RUP}{Rational Unified Process}
\newacronymeng{rup}{RUP}{Proceso Unificado de Rational}
\newacronym{gui-en}{GUI}{Graphical User Interface}
\newacronymeng{gui}{GUI}{Interfaz Gr�fica de Usuario}

\addbibresource{bib/ref.bib}

\begin{document}
	\maketitle
	\introduction
El campo de desarrollo que proyecta el escenario de la informaci�n y las comunicaciones en los �ltimos tiempos ha demostrado ser un catalizador por excelencia en el inicio del siglo XXI, marcando huellas en la Industria del Software que conducen a perfeccionar espacios educativos, did�cticos y cient�ficos, estableciendo dis�miles tendencias. Modernizar el espacio en que el hombre vive y convive es una de las proyecciones que caracteriza la revoluci�n cient�fica concentrando su actividad fundamental en la fabricaci�n de software, que hace m�s placentera, exitosa y prometedora las m�ltiples tareas a las que convoca enfrentar.
\\
\\
El hombre, en cualquier parte del mundo, se rige por reglas y normativas asignadas al control de la
sociedad, para poner freno a todas las infracciones cometidas por este. Desde los comienzos de la
humanidad fue necesario el reconocimiento de un conjunto de leyes que regularan ciertas manifestaciones del hombre en el entorno. La violaci�n de estos c�digos trae consigo una infracci�n por la que el Estado como garante o las instituciones representantes deben hacer cumplir. \citep{acanda2018cdis}
\\
\\
En Cuba, la Constituci�n de la Rep�blica de Cuba de 19871, establece las pautas fundamentales a seguir por todos los cubanos. Las instituciones en el pa�s se administran haciendo cumplir un conjunto de reglas y pol�ticas determinadas para su accionar; donde espec�ficamente en el caso del \ac{mes} , se rige por el reglamento disciplinario para los estudiantes de la educaci�n superior, puesto en vigor mediante la Resoluci�n No. 240 del a�o 20073.
La \ac{uci}, ya que pertenece al \ac{mes}, est� sujeta a cumplir este reglamento, lo cual conlleva a que si un estudiante o trabajador de la entidad incumple con alg�n o algunos de los art�culos descritos en su cuerpo legal, este debe ser procesado por una \gls{comisiond}\ en su car�cter de �rgano designado por una facultad para procesar a un estudiante o trabajador sancionado.
La gesti�n de las comisiones disciplinarias en una facultad docente posee una gran importancia debido a sus implicaciones legales.

Teniendo en cuenta la situaci�n problem�tica anteriormente descrita, se plantea como problema de investigaci�n la siguiente interrogante: 
�C�mo contribuir con informatizaci�n del proceso de comisi�n disciplinaria en la Facultad 4? 
Por lo cual el \textbf{objeto de estudio} es el proceso de comisi�n disciplinaria. 
Por tanto, el \textbf{campo de acci�n} est� enmarcado en los procesos de gesti�n de resoluciones decanales, denuncias, casos disciplinarios y declaraciones en la Facultad 4. 
Se define como \textbf{objetivo general}: Desarrollar un sistema de gesti�n para el proceso de comisi�n disciplinaria en la Facultad 4. 
A partir del objetivo general se derivan los siguientes \textbf{objetivos espec�ficos}: 
Analizar elementos te�ricos y principales tendencias del desarrollo de sistemas en la actualidad, espec�ficamente en el campo de sistemas de gesti�n de la informaci�n.
Definir las tecnolog�as y herramientas necesarias para el desarrollo de la propuesta de soluci�n. 
Implementar las funcionalidades de la propuesta de soluci�n. 
Evaluar la propuesta de soluci�n. 
Como Hip�tesis se plantea: El desarrollo de una aplicaci�n inform�tica de gesti�n para el proceso de comisi�n disciplinaria en la Facultad 4 contribuir� a la informatizaci�n del proceso de comisi�n disciplinaria en la Facultad 4. 
Se define como variable independiente: m�dulo de procesamiento estad�stico de informaci�n y como  variable dependiente: proceso de apoyo a la toma de decisiones.

\section*{M�todos cient�ficos empleados en la investigaci�n}
\subsection*{M�todos te�ricos}
\paragraph{Hist�rico-l�gico:}
Para el estudio del desarrollo y evoluci�n de los diferentes sistemas de gesti�n de informaci�n similares, nacionales e internacionales, as� como las herramientas y tecnolog�as para el desarrollo del software, entre ellos los lenguajes de programaci�n, \gls{framework}s de desarrollo, metodolog�as y herramientas \ac{case}.
\paragraph{Anal�tico-sint�tico:}
En la realizaci�n del an�lisis de la informaci�n empleada para la investigaci�n, identificando as�, conceptos, definiciones y avances acerca de los sistemas de gesti�n de informaci�n existentes. 
\paragraph{Modelaci�n:}
Se utiliza en la modelaci�n de los diagramas dentro de la metodolog�a de desarrollo de software seleccionada para llevar a cabo la soluci�n.

%\end{description}
%\begin{description}
%\item[Hist�rico-l�gico:] Para el estudio del desarrollo y evoluci�n de los diferentes sistemas de gesti�n de informaci�n similares, nacionales e internacionales, as� como las herramientas y tecnolog�as para el desarrollo del software, entre ellos los lenguajes de programaci�n, \gls{framework}s de desarrollo, metodolog�as y herramientas \ac{case}.
%
%\item[Anal�tico-sint�tico:] En la realizaci�n del an�lisis de la informaci�n empleada para la investigaci�n, identificando as�, conceptos, definiciones y avances acerca de los sistemas de gesti�n de informaci�n existentes. 
%
%\item[Modelaci�n:] Se utiliza en la modelaci�n de los diagramas dentro de la metodolog�a de desarrollo de software seleccionada para llevar a cabo la soluci�n.
%\end{description}

\subsection*{M�todos Emp�ricos}
\paragraph{An�lisi documental:}
Se utiliz� para obtener informaci�n de las necesidades existentes en \lafac de acuerdo a lo establecido en \elreglamento. 
%\begin{description}
%\item[Observaci�n:] Se utiliz� para obtener informaci�n de las necesidades existentes en la Facultad 4. 
%\end{description}


\section*{Estructura}
El presente trabajo de diploma cuenta con la siguiente estructura: Introducci�n, tres cap�tulos, conclusiones, recomendaciones, glosario y, por �ltimo, referencias bibliogr�ficas. A continuaci�n, se presenta un resumen de las diferentes tem�ticas que se abordan en los cap�tulos. 
\subsection*{Estructura capitular}

\paragraph{Cap�tulo 1: Fundamentaci�n Te�rica:} En este cap�tulo se exponen los elementos te�ricos utilizados en la investigaci�n, describiendo adem�s las tecnolog�as, metodolog�as, herramientas y el lenguaje de programaci�n utilizado en el desarrollo de la soluci�n, as� como los principales conceptos involucrados, para una mejor comprensi�n. 
\paragraph{Cap�tulo 2: Propuesta de soluci�n:} En este cap�tulo se exponen los elementos que permiten describir la propuesta de soluci�n, tales como requerimientos no funcionales e historias de usuario, para lograr un entendimiento claro del sistema a desarrollar.
\paragraph{Cap�tulo 3: Implementaci�n y pruebas:} En este cap�tulo se realiza el modelado de dise�o de las funcionalidades y la implementaci�n de la propuesta de soluci�n, donde se obtiene un recurso inform�tico que brinda los servicios requeridos por el cliente. Se describen adem�s las pruebas realizadas al sistema, para comprobar el correcto funcionamiento de la propuesta de soluci�n.

%\begin{description}
%\item[Cap�tulo 1: Fundamentaci�n Te�rica:] En este cap�tulo se exponen los elementos te�ricos utilizados en la investigaci�n, describiendo adem�s las tecnolog�as, metodolog�as, herramientas y el lenguaje de programaci�n utilizado en el desarrollo de la soluci�n, as� como los principales conceptos involucrados, para una mejor comprensi�n. 
%
%\item[Cap�tulo 2: Propuesta de soluci�n:] En este cap�tulo se exponen los elementos que permiten describir la propuesta de soluci�n, tales como requerimientos no funcionales e historias de usuario, para lograr un entendimiento claro del sistema a desarrollar. 
%
%\item[Cap�tulo 3: Implementaci�n y pruebas:] En este cap�tulo se realiza el modelado de dise�o de las funcionalidades y la implementaci�n de la propuesta de soluci�n, donde se obtiene un recurso inform�tico que brinda los servicios requeridos por el cliente. Se describen adem�s las pruebas realizadas al sistema, para comprobar el correcto funcionamiento de la propuesta de soluci�n.
%\end{description}

	\chapter{Fundamentaci�n te�rica}
\label{c:chapter1}
\section[Introducci�n]{Introducci�n del cap�tulo}

En el presente cap�tulo se engloban aspectos relacionados con el objeto de estudio definido para el problema planteado. El an�lisis de algunas metodolog�as, procedimientos, herramientas existentes para el desarrollo de sistemas web y la observaci�n de aplicaciones hom�logas; permitir� la selecci�n de las tecnolog�as adecuadas para el desarrollo del Sistema de Gesti�n par el Proceso de Comisi�n Disciplinaria y contar con un an�lisis de sistemas existentes que realizan funcionalidades similares.
Sobre la base de los elementos expuestos anteriormente se formula el siguiente problema de investigaci�n: �C�mo contribuir a la agilizaci�n del proceso de comisi�n disciplinaria en la facultad 4? Para la realizaci�n de la investigaci�n se define como objeto de estudio: el proceso de comisi�n disciplinaria qe se lleva a cabo en la facultad 4 \gls{casod} .
Para dar soluci�n al problema planteado, se define como objetivo general: desarrollar un sistema para el la gesti�n del proceso de comisi�n disciplinaria en la Facultad 4 de la \ac{uci}

Para dar cumplimiento al objetivo general antes mencionado, se dar� cumplimiento a los siguientes objetivos espec�ficos:

\begin{enumerate}
	\item Describir el estado actual de las herramientas dirigidas a la gesti�n de procesos disciplinarios en casas de altos estudios.
	\item Definir las tecnolog�as, herramientas y metodolog�a a utilizar en la implementaci�n de un sistema de gesti�n para el proceso de comisi�n disciplinaria en la facultad 4.	\item Dise�ar las funcionalidades sistema de gesti�n para el proceso de comisi�n disciplinaria en la facultad 4.
	\item Implementar y validar las funcionalidades del sistema de gesti�n para el proceso de comisi�n disciplinaria en la facultad 4
\end{enumerate}

\textbf{Hip�tesis:}
Con el desarrollo de un sistema de gesti�n para el proceso de comisi�n disciplinaria en la facultad 4 se contribuir� a la mejora del proceso de apoyo a la toma de decisiones.
Se define como \emph{variable independiente}: m�dulo de procesamiento estad�stico de informaci�n y como  \emph{variable dependiente}: proceso de apoyo a la toma de decisiones.

\section{Formulaci�n de la propuesta de soluci�n}
%	SECTION
\section{Herramientas y tecnolog�as a utilizar}

Para el desarrollo de cualquier aplicaci�n, es necesario utilizar diferentes t�cnicas como: los patrones de dise�o o las metodolog�as de desarrollo de software, adem�s del uso de distintas herramientas como los compiladores o editores de c�digos. Aunque a simple vista parezca que la selecci�n de las tecnolog�as para desarrollar aplicaciones es f�cil, es totalmente lo contrario; para su correcta selecci�n, es necesario ver el problema a resolver desde diferentes �ngulos y posibles situaciones futuras. El presente ep�grafe aborda alguna de las diferentes herramientas que dan soluci�n a la problem�tica planteada.

\subsection{Patrones de dise�o: }

Seg�n el libro Dive Into Design Patterns, los patrones de dise�o son:
\begin{quote}
	``Soluciones t�picas a problemas comunes en el desarrollo de software. Se podr�a decir que son como planos predefinidos que pueden ser adaptados para resolver problemas en el dise�o de la codificaci�n de un programa~\citep{Shevts2019}.''
\end{quote}

Normalmente se confunden con algoritmos, porque ambos conceptos describen soluciones t�picas a problemas conocidos, mientras que un algoritmo describe una serie de pasos a seguir para lograr un objetivo, un patr�n es una descripci�n de alto nivel de la soluci�n, o sea, la codificaci�n de un mismo patr�n puede ser diferente en programas distintos~\citep{Shevts2019}.

Los patrones de dise�o difieren entre ellos debido a su complejidad, el nivel de detalles necesarios y la escala del sistema que se va a implementar. Los patrones de bajo nivel son llamados idiomas y usualmente solo se aplican a un lenguaje de programaci�n. Mientras que los patrones m�s universales y de m�s alto nivel, son llamados patrones arquitect�nicos. Estos �ltimos pueden ser usados en cualquier programa independiente del lenguaje en que sea programado y adem�s pueden ser utilizados para crear la arquitectura completa de un software~\citep{Shevts2019}. 

\subsubsection{Surgimiento de los patrones:}

En un principio, no fueron llamados patrones, ni estaban agrupados, sino que fueron soluciones que se repitieron una y otra vez en el desarrollo de software. Debido a esto, Erich Gamma, Jhon Vlissides, Ralph Johnson y Richard Helm en 1995 escribieron el libro ``Design Patterns: Elements of Reusable Object-Oriented software'', libro que reun�a y clasificaba las soluciones hasta ahora utilizadas. 

El concepto de patr�n se di� a conocer en el libro ``Pattern Language: Towns, Buildings, Construction'' del autor Christopher Alexander, donde se describ�a un lenguaje natural para la construcci�n de edificios. Teniendo el libro anteriormente mencionado como base, fue que estas soluciones a problemas repetitivos fueron nombradas como patrones de dise�o de programaci�n.

Con el tiempo, estas cuatro personas pasaron a llamarse Gang of Four (Banda de los cuatro) y a su vez el nombre del libro paso a ser ``The GOF book''.

\subsubsection{Tipos de patrones}
En total el libro recoge 23 patrones, divididos en tres categor�as seg�n su intenci�n:
\begin{itemize}
	\item Patrones Creacionales:
	\begin{itemize}
		\item Provee mecanismos para la creaci�n de objetos lo que incrementa la flexibilidad y la reutilizaci�n de c�digo existente.
	\end{itemize}
	\item Patrones Estructurales:
	\begin{itemize}
		\item Explica como ensamblar objetos y clases dentro de largas estructuras, mientras que la estructura se mantiene flexible y eficiente.
	\end{itemize}
	\item Patrones de Comportamiento:
	\begin{itemize}
		\item Se encarga de la comunicaci�n eficiente y la asignaci�n de responsabilidades entre los objetos.
	\end{itemize}
\end{itemize}

\subsection{Metodolog�as para el desarrollo de software:}

Los softwares han sido parte de la vida cotidiana de la humanidad desde hace muchos a�os, pero su desarrollo comenz� de una manera muy desorganizada, ya que se basaba en actividades de codificar y arreglar los errores. Los softwares en su mayor�a eran creados sin seguir una l�nea de actividades, por lo que la estructura de estos pod�a variar frecuentemente. Pero como todo en la vida, llega un momento en que hay que organizar la forma en que se hacen las cosas, por lo que surgi� una alternativa llamada metodolog�a. Esta dictaba un proceso bien estructurado para la creaci�n de software, lo cual hac�a el desarrollo de estos m�s predictibles y eficientes.\\

Primero surgieron las metodolog�as tradicionales poseedoras de un plan de trabajo extenso, debido a la documentaci�n de los requerimientos del software, seguidos por la especificaci�n de la arquitectura y una representaci�n de alto nivel del software a desarrollar. Debido a la cantidad de trabajo a realizar, las metodolog�as tradicionales pasaron a ser conocidas como metodolog�as pesadas. Mayormente se utilizaban en software con un alto impacto, ya sea para la vida o la sociedad. Pero los proyectos que no pose�an un impacto tan notable, tambi�n deb�an utilizar las metodolog�as pesadas, por lo que el trabajo era muy lento y lleno de dificultades.\\

En respuesta al trabajo excesivo y minucioso en proyectos sin grandes repercusiones en la sociedad, nacieron las metodolog�as �giles, que fueron destinadas al desarrollo de software mediante la interacci�n con el cliente. Cambios en la estructura del proyecto de forma m�s seguida y un desarrollo r�pido, eran las caracter�sticas que m�s difer�an del prop�sito principal de las metodolog�as pesadas \citep{Awad2005}.\\


Dentro de las metodolog�as �giles se encuentran:

	\paragraph{ \ac{xp}:} Se caracteriza por los ciclos de desarrollo cortos, el incremento de los planes para el desarrollo del software, adem�s de la retroalimentaci�n que se establece con el cliente.
	\paragraph{ SCRUM:}
	 Describe la forma en que los miembros de equipo deben trabajar para poder obtener un sistema flexible en un entorno que var�a de manera constante.
 \paragraph{Agile Lite:}
		``(...)puede ser aplicada a cualquier proyecto, asumiendo que el trabajo a realizar se pueda dividir en peque�as acciones.'' \citep{AgileLiteHumanos}
		Utiliza ciclos de desarrollos cortos.\\


Con la ayuda de la informaci�n anteriormente expuesta, ya es hora de seleccionar una metodolog�a para el desarrollo del \cdis, pero antes, se deben poner en una balanza las caracter�sticas del mismo:


\input{tables/methodology-selection}

Seg�n la tabla anterior, el software debe ser desarrollado siguiendo las pautas de las metodolog�as �giles, ya que cumple 11 de los 13 aspectos necesarios para decantarse por las mismas.

Dentro de las posibles metodolog�as �giles a seleccionar se encuentran \ac{xp}, Scrum y Agile Lite. Se decide utilizar \ac{xp}.

\input{subsections/xp}
\section{Historias de usuario}

\begin{userstory}
	\storyname{Crear resoluci�n}
	\storyuser{Specialist}
	\storyiter{1}
	\storypriority{ High }
	\storyrisk{ Low }
	\storypoints{0.8}
	\storyprogrammer{John Doe }
	\storydescription{\lipsum[1]}
	\storyobservation{\lipsum[1]}
\end{userstory}

	\chapter{Descripci�n y dise�o}

\label{c:chapter2}
 \section{Introducci�n}
 
 El presente cap�tulo abordan las particularidades del sistema de gesti�n a desarrollar. Para registrar las principales caracter�sticas se hace uso de los \ac{rf} y \ac{rnf}. Estos describen las funcionalidades y atributos de calidad que debe poseer el software. 
 	
 En el cap�tulo \ref{c:chapter1}, se seleccion� la metodolog�a XP como gu�a para el desarrollo del software; por lo tanto, se utilizan las \ac{hu} como herramienta para una descripci�n detallada de los \ac{rf} y la confecci�n del plan de iteraciones. Mediante el uso de este �ltimo, se proceder� a la estimaci�n del tiempo requerido para la culminaci�n del desarrollo del sistema y, con el uso de patrones de dise�o, se facilitar� la posterior descripci�n de las tarjetas \ac{crc}.
\section{Requisitos}\label{s:req}

\subsection{Requisitos funcionales}
En ingenier�a de software, los \ac{rf} definen un sistema o sus componentes; describen la funci�n que un software debe realizar, ya sean c�lculos, manipulaci�n de datos, procesos de negocios, entre otros.
Ayudan adem�s a capturar los comportamientos planificados para un sistema, este comportamiento puede ser expresado como una funci�n, servicio o tarea que un software debe realizar \citep{pressman2005software}. A continuaci�n, se exponen los diferentes \ac{rf} planteados por el usuario:
\begin{RF}
    \item Autenticar Usuario
    \item Asignar rol a usuario
    \\
    \item Listar denuncias %w
    \item Crear denuncia
    \item Modificar denuncia
    \item Eliminar denuncia
    \item Buscar denuncia
    %\item Exportar tabla de denuncias
    \item Exportar denuncia
    %\item Imprimir listado de denuncias
    \\
    \item Listar resoluciones decanales %w
    \item Crear resoluci�n decanal %w
    \item Modificar resoluci�n decanal
    \item Eliminar resoluci�n decanal %w
    \item Exportar resoluci�n decanal
    \\
    \item Listar comisiones %w
    \item Crear comisi�n %w
    \item Modificar comisi�n
    \item Eliminar comisi�n %w
    \item Buscar comisi�n %w
    \\
    \item Listar declaraciones
    \item Crear declaraci�n
    \item Modificar declaraci�n
    \item Eliminar declaraci�n
    \item Buscar declaraci�n
    % \item Exportar declaraci�n
    %\item Exportar tabla de declaraciones
    \\
    \item Crear caso disciplinario
    \item Listar casos disciplinarios
    \item Modificar caso disciplinario
    \item Buscar caso disciplinario
    \\
    \item Exportar resoluci�n de caso
    \item Exportar conclusi�n de la comisi�n
    \\
    \item Listar roles
    \item Crear rol
    \item Modificar rol
    \item Eliminar rol
    \item Buscar rol
    %\item Exportar tabla de usuarios
\end{RF}

\subsection{Requisitos no funcionales}
Los \ac{rnf}, como su nombre sugiere, son aquellos requerimientos que no se refieren directamente a las funciones espec�ficas que proporciona el sistema, sino a las propiedades emergentes de este como la fiabilidad, el tiempo de respuesta y la capacidad de almacenamiento. De forma alternativa definen las restricciones del sistema \citep{sommerville2011software}.\\
Los requisitos no funcionales de la aplicaci�n a desarrollar son:
\\
\noindent Usabilidad:
\begin{RNF}
    \item La \ac{gui} de la aplicaci�n cliente debe ser intuitiva y f�cil de usar por cualquier persona con edad laboral y con conocimientos b�sicos de trabajo con software, de acuerdo a est�ndares modernos de dise�o de \ac{gui}.
\end{RNF}
Hardware y software:
\begin{RNF}[resume]
    \item El software debe poder funcionar en un servidor con 4GB de \ac{ram} y 10GB de almacenamiento libre.
    \item El cliente del sistema debe ser compatible con las �ltimas versiones estables de los principales navegadores web: Google Chrome, navegadores basados en \gls{chromium}, Mozilla Firefox y Safari.
    %TODO Seguridad?  \item El usuario debe poder acceder a las funcionalidades del sistema s�lo despu�s de estar autenticado.
    % \item El usuario debe poder acceder solamente a las funcionalidades permitidas de acuerdo a su rol en el sistema.
\end{RNF}
Dise�o e implementaci�n:
\begin{RNF}[resume]
    \item El sistema debe ser accesible a trav�s de una aplicaci�n web cliente.
    \item Se deben usar lenguajes de programaci�n y herramientas com amplio soporte y estabilidad.
\end{RNF}

\section{Historias de usuario}
\label{s:hu}
Las HU ser�n representadas mediante tablas divididas por las siguientes secciones:
\begin{description}

\item N�mero: n�mero de la historia de usuario incremental en el tiempo;
\item Nombre de historia de usuario: el nombre de la historia de usuario ser�a para identificarlas mejor
entre los desarrolladores y el cliente;
\item[Usuario] El usuario que est� involucrado en el desarrollo de la HU;
\item[Iteraci�n asignada] N�mero de la iteraci�n;
\item[Prioridad en negocio] 
\begin{itemize}
\item Las historias de usuarios que son de funcionalidades imprescindibles en el desarrollo del sistema tienen prioridad alta;
\item Las historias de usuarios que son de funcionalidades que debe de tener el sistema, pero que
no son necesarias para su funcionamiento, tienen prioridad media;
\item Las historias de usuarios que son de funcionalidades auxiliares y que son independientes del
sistema, tienen prioridad baja.
\end{itemize}
\item Riesgo en desarrollo:
\begin{itemize}
\item Las historias de usuarios que, en caso de tener alg�n error de implementaci�n, puedan afectar
la disponibilidad del sistema, tienen riesgo de desarrollo alto;
\item Las historias de usuarios que puedan presentar errores y retrasan la entrega de la versi�n,
tienen riesgo de desarrollo medio;
\item Las historias de usuario que puedan presentar errores, pero estos son tratados con facilidad y
no afectan en desarrollo del proyecto, tienen riesgo de desarrollo bajo.
\end{itemize}
\item[Puntos estimados] tiempo estimado que se demorar� el desarrollo de la HU;
\item[Descripci�n] breve descripci�n de la HU;
\item[Observaciones] se�alamiento o advertencia del sistema;
\item[Prototipo de interfaz] prototipo de interfaz si aplica.
\end{description} \citep{Joskowicz2008}


Los t�tulos de las HU generadas son:
\begin{enumerate}[label=HU \arabic*:]
\item  Autenticar usuario.
\item  Gestionar usuario.
\item  Mostrar reportes al asesor.
\item  Gestionar denuncias.
\item  Mostrar clasificaci�n de las faltas.
\item  Gestionar comisiones disciplinarias.
\item  Gestionar expediente disciplinario.
\item  Gestionar dict�menes.
\item  Gestionar pr�rroga.
\item  Gestionar circunstancias modificativas.
\item  Exportar a PDF.
\end{enumerate}
A continuaci�n se presentan las historias de usuarios identificadas en la investigaci�n
\begin{userstory}
	\storyname{Crear denuncia}
	\storyuser{\userY}
	\storyiter{1}
	\storypriority{ High }
	\storyrisk{ Low }
	\storypoints{0.8}
	\storyprogrammer{\authorA }
	\storydescription{}
	\storyobservation{\lipsum[1]}
\end{userstory}

\begin {effortestimation}
 \addentry [1]{ Listar denuncias }{0.2}
 \addentry [1]{ Crear denuncia }{0.4}
 \addentry [1]{ Modificar denuncia }{0.4}
 
 \addentry [1]{ Listar comisiones}{0.2}
 \addentry [1]{ Crear comisi�n }{0.4}
 \addentry [1]{ Modificar comisi�n }{0.4}
 
 \addentry [1]{ Listar resoluciones }{0.2}
 \addentry [1]{ Crear resoluci�n }{0.4}
 \addentry [1]{ Modificar resoluci�n }{0.4}
 
 \addentry [2]{ Modificar permisos de usuario }{0.2}
 
 \addentry [3]{ Exportar resoluci�n }{0.6}
 \addentry [3]{ Exportar expediente }{0.6}
\end{effortestimation}
%\geniterationplan

	\chapter{Validaci�n de la soluci�n}\label{c:chapter3}
\subsection*{Introducci�n del cap�tulo} El presente cap�tulo describe la etapa de validaci�n de la soluci�n. Para ello, se exponen las diferentes pruebas realizadas durante el desarrollo del sistema que llevan al lanzamiento de una soluci�n que resuelva la situaci�n problem�tica planteada.
\section{Pruebas}
Uno de los pilares de \ac{xp} es el proceso de pruebas. Esta metodolog�a de desarrollo anima a probar constantemente tanto como sea posible. Esto permite aumentar la calidad de los sistemas reduciendo el n�mero de errores no detectados y disminuyendo el tiempo transcurrido entre la aparici�n de un error y su detecci�n. Tambi�n permite aumentar la seguridad de evitar efectos colaterales no deseados a la hora de realizar  dise�ada por los programadores, y pruebas de aceptaci�n o pruebas funcionales destinadas a evaluar si al final de una iteraci�n se consigui� la funcionalidad requerida dise�adas por el cliente final \citep{gutierrez2006pruebas}.
Con el objetivo de comprobar que los sistemas desarrollados funcionan de acuerdo a las especificaciones descritas por el cliente, se realizaron diferentes pruebas teniendo en cuenta las caracter�sticas de los m�dulos.
 
\subsection{Pruebas unitarias}
Las pruebas unitarias o unittesting son una forma de comprobar que un fragmento de c�digo funciona correctamente.Son peque�os tests que validan el comportamiento de un objeto y la l�gica \citep{gutierrez2006pruebas}. \ac{xp} plantea la realizaci�n de pruebas unitarias continuas, frecuentemente repetidas y automatizadas, incluyendo pruebas de regresi�n, y aconseja escribir el c�digo de la prueba antes de la codificaci�n \citep{kniberg2007scrum}.\\
Para la aplicaci�n de las pruebas unitarias a la soluci�n se emple� el framework JUnit el cual permite realizar la ejecuci�n de clases Java de manera controlada, para poder evaluar si el funcionamiento de cada uno de los m�todos de la clase se comporta como se espera. Es decir, en funci�n de alg�n valor de entrada se eval�a el valor de retorno esperado; si la clase cumple con la especificaci�n, entonces JUnit devolver� que el m�todo de la clase pas� exitosamente la prueba; en caso de que el valor esperado sea diferente al que regres� el m�todo durante la ejecuci�n, JUnit devolver� un fallo en el m�todo correspondiente.
\section*{Conclusiones del cap�tulo}
Al finalizar el presente cap�tulo se arriba a las siguientes conclusiones parciales:

\begin{itemize}
    \item Definir una estrategia de pruebas contribuy� a un temprano descubrimiento de errores y no conformidades en la propuesta de soluci�n.
    \item Luego de cada cambio significativo en el sistema, se realizaron pruebas unitarias automatizadas a las principales funcionalidades, exigi�ndose un 100\% de efectividad para las mismas. Los resultados fueron buenos. Permitieron garantizar estabilidad durante el desarrollo y sentar las bases para las pruebas de sistema.
    \item Las pruebas de sistema permitieron validar la soluci�n, y verificar que se cumplieron los requisitos identificados por el cliente y el equipo de desarrollo para cada iteraci�n.
\end{itemize}
	\conclusions
Tras finalizar la presente investigaci�n se puede concluir que se desarrollaron todas las tareas que tributan al cumplimiento de los objetivos propuestos, resaltando que:
\begin{itemize}
    \item El estudio de la bibliograf�a y an�lisis de los sistemas existentes para la gesti�n del \proceso\ demostr� la necesidad del desarrollo de un nuevo sistema inform�tico para llevar a cabo la gesti�n del mismo en \lafac.
    \item Durante la planificaci�n y el dise�o de la soluci�n se generaron los artefactos propuestos por la metodolog�a de desarrollo seleccionada, lo cual permiti� facilitar la implementaci�n de las funcionalidades definidas.
    \item La implementaci�n de la propuesta de soluci�n contribuye a la gesti�n del \proceso\ en \lafac.
    \item Las pruebas de software realizadas permitieron garantizar un correcto funcionamiento de la herramienta de acuerdo a lo planificado en cada entrega, as� como el cumplimiento de las necesidades y requisitos del cliente.
\end{itemize}
%Se cumple el objetivo general de desarrollar un \cdis luego de haber analizado elementos te�ricos y principales tendencias del desarrollo de sistemas en la actualidad, espec�ficamente en el campo de sistemas de gesti�n de la informaci�n, definido las tecnolog�as y herramientas necesarias para el desarrollo de la propuesta de soluci�n, implementado y evaluado las funcionalidades de la misma.
% Al finalizar la presente investigaci�n se conclye que el proceso de comisi�n disciplinaria que se lleva a cabo en la Facultad 4 de la Universidad de las Ciencias Inform�ticas, el cual se realiza de manera manual puede ejecutarse ahora de manera semi-automatizada a trav�s del Sistema para la Gesti�n del Proceso de Comisi�n Disciplinaria en la Facultad 4 de la Universidad de las Ciencias Inform�ticas. Siendo este, adem�s, una soluci�n escalable y adaptable, lo cual facilitar� la implementaci�n de nuevas funcionalidades.

	\suggestions
A partir de los resultados obtenidos se recomienda:
\begin{itemize}
    \item Implementar un m�dulo de an�lisis estad�stico.
    \item Incluir nuevas funcionalidades como manejo de apelaciones y comisiones mixtas.
\end{itemize}
	\appendixes
\begin{addendum}
    \chapter{Artefactos ingenieriles}
    \section{Historias de usuario}
\label{a:hu}

\begin{userstory}
	\storyname{Autenticar usuario}
	\storyuser{Usuario}
	\storyiter{1}
	\storypriority{ Alta }
	\storyrisk{ Bajo }
	\storypoints{\hpa}
	\storyprogrammer{\authorA }
	\storydescription{Como usuario quiero ingresar al sistema para realizar las acciones que me corresponden.}
	\storyobservation{}
	\storyinterface{
		Formulario de autenticaci�n:\\
		\includegraphics[scale=0.5]{images/prototypes/cdis-login-capture.png}\\
		Informaci�n del usuario autenticado:\\
		\includegraphics[scale=0.5]{images/prototypes/cdis-loggedin-capture.png}
	}
\end{userstory}

\begin{userstory}
	\storyname{Asignar rol a usuario}
	\storyuser{Administrador}
	\storyiter{1}
	\storypriority{ Alta }
	\storyrisk{ Bajo }
	\storypoints{\hpa}
	\storyprogrammer{\authorA }
	\storydescription{Como administrador quiero asignar un rol a un usuario para que pueda realizar las acciones que le corresponden.}
	\storyobservation{}
	\storyinterface{
		% Formulario de asignaci�n de rol:\\
		% \includegraphics[scale=0.5]{images/prototypes/cdis-assignrole-capture.png}
	}
\end{userstory}



\begin{userstory}
	\storyname{Realizar denuncia}
	\storyuser{Usuario}
	\storyiter{1}
	\storypriority{ Alta }
	\storyrisk{ Bajo }
	\storypoints{\hpc}
	\storyprogrammer{\authorA }
	\storydescription{Como usuario quiero realizar denuncias a trav�s del sistema para que sean registradas en el mismo y entregadas autom�ticamente a las autoridades competentes.}
	\storyobservation{}
	\storyinterface{
		Bot�n de acci�n:\\
		\includegraphics[scale=0.4]{images/prototypes/cdis-create-capture.png}\\
		Formulario:\\
		\includegraphics[scale=0.4]{images/prototypes/cdis-create-denuncia-capture.png}}

\end{userstory}
\begin{userstory}
	\storyname{Consultar denuncias}
	\storyuser{Usuario}
	\storyiter{1}
	\storypriority{ Alta }
	\storyrisk{ Bajo }
	\storypoints{\hpb}
	\storyprogrammer{\authorA }
	\storydescription{Como usuario quiero consultar denuncias a las que tengo acceso para conocer su estado.}
	\storyobservation{}
	\storyinterface{}
\end{userstory}

\begin{userstory}
	\storyname{Listar resoluciones de casos}
	\storyuser{Decano}
	\storyiter{1}
	\storypriority{ Alta }
	\storyrisk{ Bajo }
	\storypoints{\hpc}
	\storyprogrammer{\authorA }
	\storydescription{Como Decano quiero listar las resoluciones de los casos disciplinarios para poder consultarlas en cualquier momento}
	\storyobservation{}
	\storyinterface{}
\end{userstory}
\begin{userstory}
	\storyname{Crear resoluci�n}
	\storyuser{Decano}
	\storyiter{1}
	\storypriority{ Alta }
	\storyrisk{ Bajo }
	\storypoints{\hpc}
	\storyprogrammer{\authorA }
	\storydescription{Como decano quiero redactar resoluciones en la aplicaci�n para tener acceso las comisiones definidas en ellas mediante el sistema.}
	\storyobservation{}
	\storyinterface{}
\end{userstory}
\begin{userstory}
	\storyname{Consultar denuncias}
	\storyuser{Decano}
	\storyiter{1}
	\storypriority{ Alta }
	\storyrisk{ Bajo }
	\storypoints{\hpr}
	\storyprogrammer{\authorA }
	\storydescription{Como decano quiero consultar las denuncias realizadas para poder asignarlas a las comisiones disciplinarias.}
	\storyobservation{}
	\storyinterface{}
\end{userstory}
\begin{userstory}
	\storyname{Crear caso disciplinario}
	\storyuser{Decano}
	\storyiter{1}
	\storypriority{ Alta }
	\storyrisk{ Bajo }
	\storypoints{\hpc}
	\storyprogrammer{\authorA }
	\storydescription{Como decano quiero crear casos disciplinarios para que se registre toda la informaci�n generada durante la investigaci�n de la indisciplina.}
	\storyobservation{}
	\storyinterface{}
\end{userstory}
\begin{userstory}
	\storyname{Exportar resoluci�n de caso}
	\storyuser{Decano}
	\storyiter{3}
	\storypriority{ Alta }
	\storyrisk{ Bajo }
	\storypoints{\hpc}
	\storyprogrammer{\authorA }
	\storydescription{Como decano quiero exportar la resoluci�n de un caso disciplinario para poder imprimirla.}
	\storyobservation{}
	\storyinterface{}
\end{userstory}
\begin{userstory}
	\storyname{Exportar conclusi�n de la comisi�n}
	\storyuser{Presidente de comisi�n}
	\storyiter{3}
	\storypriority{ Alta }
	\storyrisk{ Bajo }
	\storypoints{\hpe}
	\storyprogrammer{\authorA }
	\storydescription{Como Presidente de comisi�n quiero exportar la conclusi�n de la comisi�n para poder imprimirla.}
	\storyobservation{}
	\storyinterface{}
\end{userstory}





% INFO FOR THE TASK \begin{userstory}
% 	\storyname{Realizar denuncia}
% 	\storyuser{\authorA}
% 	\storyiter{1}
% 	\storypriority{ Alta }
% 	\storyrisk{ Bajo }
% 	\storypoints{\hpc}
% 	\storyprogrammer{\authorA }
% 	\storydescription{El sistema debe permitir registrar una denuncia en el sistema. Para ello se le debe brindar al usuario un formulario en el que ingresar los datos de la nueva denuncia y botones para registrarla en el sistema o cancelar la acci�n.
% 		Los datos a ingresar en el formulario deben ser:
% 		\begin{itemize}
% 			\item Estudiantes involucrados
% 			\item Fecha de ocurrencia de los hechos descritos en la denuncia
% 			\item Hora de ocurrencia de los hechos descritos en la denuncia
% 			\item Descripci�n
% 			\item Adjuntos (\textit{opcional})
% 		\end{itemize}}
% 	\storyobservation{Para registrar una denuncia tambi�n se necesita la informaci�n del denunciante y la fecha de la denuncia, pero se debe tomar el usuario autenticado como denunciante y la fecha de creaci�n de la denuncia en el sistema y no es necesario mostrar campos para esa informaci�n en el formulario }
% 	\storyinterface{
% 		Bot�n de acci�n:\\
% 		\includegraphics[scale=0.4]{images/prototypes/cdis-create-capture.png}\\
% 		Formulario:\\
% 		\includegraphics[scale=0.4]{images/prototypes/cdis-create-denuncia-capture.png}}

% \end{userstory}
% \begin{userstory}
% 	\storyname{Crear comisi�n disciplinaria}
% 	\storyuser{\authorA}
% 	\storyiter{1}
% 	\storypriority{ Alta }
% 	\storyrisk{ Medio }
% 	\storypoints{1}
% 	\storyprogrammer{\authorA }
% 	\storydescription{El sistema debe permitir registrar una comisi�n disciplinaria. Para ello se le debe brindar al usuario un formulario en el que ingresar los datos de la nueva comisi�n disciplinaria y botones para registrarla en el sistema o cancelar la acci�n.
% 	Los datos a ingresar en el formulario deben ser:
% 		\begin{itemize}
% 		\item Presidente de comisi�n
% 		\item Secretario de comisi�n
% 		\end{itemize}}
% 	\storyobservation{Para registrar una comisi�n tambi�n se necesita la informaci�n de la resoluci�n a la que pertenece, pero, si se permite crear las comisiones en el formulario que se brinda para registrar o modificar una resoluci�n decanal, se debe tomar la resoluci�n a crear o editar como resoluci�n a la que pertenece la comisi�n creada}
% 	\storyinterface{
% 		\includegraphics{images/prototypes/cdis-create-capture.png}

% 		}

% \end{userstory}
% \begin{userstory}
% 	\storyname{Crear resoluci�n}
% 	\storyuser{\authorA}
% 	\storyiter{1}
% 	\storypriority{ Alta }
% 	\storyrisk{ Bajo }
% 	\storypoints{\hpc}
% 	\storyprogrammer{\authorA}
% 	\storydescription{El sistema debe permitir registrar una resoluci�n decanal en el sistema. Para ello se le debe brindar al usuario un formulario en el que ingresar los datos de la nueva resoluci�n decanal y botones para registrarla en el sistema o cancelar la acci�n.
% 		Los datos a ingresar en el formulario deben ser:
% 		\begin{itemize}
% 			\item Curso. Puede ser un n�mero correspondiente al a�o o una cCadena de caracteres que lo identifica.
% 			\item Comisiones disciplinarias
% 		\end{itemize}}
% 	\storyobservation{Para registrar una resoluci�n decanal tambi�n se necesita la informaci�n del decano que la emite, para ello se deben tomar los datos del usuario autenticado como decano emisor de la misma.}
% 	\storyinterface{
% 		Bot�n de acci�n:\\
% 		\includegraphics[scale=0.4]{images/prototypes/cdis-create-capture.png}\\
% 		Formulario:\\
% 		\includegraphics[scale=0.4]{images/prototypes/cdis-create-resolucion-capture.png}
% 	}
% \end{userstory}

    \section{Tarjetas CRC}
\label{add:crc}
\begin{crccard}
	\crcclass{CasoController}
	\crcresp{
		\begin{itemize}
			\item crearCaso: Crear el expediente asociado a cada estudiante denunciado, adem�s de crear el caso que asocia una comisi�n con una denuncia
			\item listar: Mostrar el listado de casos
			\item modificar: Editar un caso requerido pudiendo definir si se encuentra activo o no y los vocales que se van a a�adir tambi�n, adem�s de modificar la denuncia para que se sepa que ya est� siendo atendida
		\end{itemize}
	}
	\crccolab{
		DenunciaService\\
		ComisionService\\
		CasoService\\
		Convertidor\\
		SesionDetails\\
		ValidatorCaso\\
		DenunciaUsuarioService\\
		ExpedienteService\\
		RolService\\
		CasoUsuarioService\\
		JsonCrearCaso\\
		UsuarioRol\\
		JsonModificarCaso\\
		E\_Caso\\
		Caso\\
		CasoUsuario\\
		Denuncia\\
		DenunciaUsuario
		Mensaje\\
		Expediente\\
	}
\end{crccard}

\begin{crccard}
	\crcclass{DeclaracionController}
	\crcresp{
		\begin{itemize}
			\item crear: Crear la declaraci�n que permite dar constancia de las evidencias y recopilaci�n de informaci�n sobre el caso
			\item listar: Mostrar las declaraciones hechas para el caso que se est� atendiendo
			\item modificar: Editar la declaracion en cuesti�n, esta opci�n solo la tiene el usuario due�o de la declaraci�n
			\item borrar: Eliminar la declaraci�n
		\end{itemize}
	}
	\crccolab{
		DeclaracionService\\
		CasoService\\
		UsuarioService\\
		SesionDetails\\
		Convertidor\\
		ValidatorDeclaracion\\
		RolService\\
		JsonBorrarDeclaraciones\\
		JsonCrearDeclaracion\\
		JsonModificarDeclaracion\\
		E\_Declaracion\\
		Mensaje\\
		Declaracion\\
		DeclaracionPK\\
		Usuario\\
	}
\end{crccard}

\begin{crccard}
	\crcclass{RolController}
	\crcresp{
		\begin{itemize}
			\item crear: Crear un nuevo rol y asignarle un listado de permisos
			\item listar: Mostrar el listado de roles existentes
			\item listarPermisos: Mostrar el listado de permisos asignados a un determinado rol
			\item modificar: Modificar un rol
			\item borrar: Eliminar un rol
		\end{itemize}
	}
	\crccolab{
		CreateRoles\\
		JsonBorrarRol\\
		JsonCrearRol\\
		JsonModificarRol\\
		E\_Permiso\\
		E\_Rol\\
		ValidatorRol\\
		Convertidor\\
		Permiso\\
		Rol\\
		Mensaje\\
		PermisoService\\
		RolService\\
	}
\end{crccard}

\begin{crccard}
	\crcclass{UsuarioController}
	\crcresp{
		\begin{itemize}
			\item crear: Crear un nuevo usuario en la base de datos del sistema CDIS
			\item listar: Mostrar el listado de usuarios
			\item borrar: Eliminar un usuario
			\item login: Autenticar un usuario en el sistema
		\end{itemize}
	}
	\crccolab{
		ClienteAutenticacionUCIWSDL\\
		JsonBorrarUsuarios\\
		JsonCrearUsuario\\
		JsonJwtDto\\
		JsonLoginUsuario\\
		E\_Usuario\\
		ValidatorUsuario\\
		Convertidor\\
		Rol\\
		Usuario\\
		Mensaje\\
		SesionDetails\\
		RolNombre\\
		JwtProvider\\
		RolService\\
		UsuarioService\\
		AutenticarUsuarioResponse\\
		Persona\\
	}
\end{crccard}

\begin{crccard}
	\crcclass{CasoService}
	\crcresp{
		\begin{itemize}
			\item la funci�n de esta clase es ser intermediaria con la clase repositorio de la base de datos modificando los m�todos de gesti�n de datos preestablecidos
			\item save: Salva una instancia de la entidad
			\item findAll: Devuelve un listado de instancias de la entidad
			\item findByDenuncia: Devuelve una instancia de la entidad
			\item deleteAll: Elimina un listado de instancias de la entidad
			\item existsByDenuncia: Confirma si existe la instancia de la entidad
			\item deleteByDenuncia: Elimina una instancia de la entidad
			\item findAllByAbierto: Devuelve un listado de instancias de la entidad filtrando si estan abiertas o no
		\end{itemize}
	}
	\crccolab{
		Caso\\
		Denuncia\\
		CasoRepo\\
	}
\end{crccard}

\begin{crccard}
	\crcclass{CasoUsuarioService}
	\crcresp{
		\begin{itemize}
			\item la funci�n de esta clase es ser intermediaria con la clase repositorio de la base de datos modificando los m�todos de gesti�n de datos preestablecidos
			\item save: Salva una instancia de la entidad
			\item findAllByUsuario: Devuelve un listado de instancias de la entidad filtrando por usuario
		\end{itemize}
	}
	\crccolab{
		CasoUsuario\\
		Usuario\\
		CasoUsuarioRepo\\
	}
\end{crccard}

\begin{crccard}
	\crcclass{ComisionService}
	\crcresp{
		\begin{itemize}
			\item la funci�n de esta clase es ser intermediaria con la clase repositorio de la base de datos modificando los m�todos de gesti�n de datos preestablecidos
			\item save: Salva una instancia de la entidad
			\item findAll: Devuelve un listado de instancias de la entidad
			\item delete: Elimina una instancia de la entidad
			\item findById: Devuelve una instancia de la entidad filtrando por id
			\item deleteAll: Elimina un listado de instancias de la entidad
			\item existsById: Confirma si existe una instancia de la entidad por el id
			\item findAllByResolucion: Devuelve un listado de instancias de la entidad filtrando por resolucion
			\item deleteById: Elimina una instancia de la entidad filtrando por id
		\end{itemize}
	}
	\crccolab{
		Comision\\
		Resolucion\\
		ComisionRepo\\
	}
\end{crccard}

\begin{crccard}
	\crcclass{ComisionUsuarioService}
	\crcresp{
		\begin{itemize}
			\item la funci�n de esta clase es ser intermediaria con la clase repositorio de la base de datos modificando los m�todos de gesti�n de datos preestablecidos
			\item save: Salva una instancia de la entidad
			\item findAllByComision: Devuelve un listado de instancias de la entidad filtrando por comision
			\item delete: Elimina una instancia de la entidad
			\item findByComisionUsuarioPK: Devuelve una instancia de la entidad filtrando por comisionUsuarioPK
			\item saveAll: Salva un listado de instancias de la entidad
			\item deleteAll: Elimina un listado de instancias de la enttidad
			\item deleteByComisionUsuarioPK: Elimina una instancia de la entidad filtrando por comisionUsuarioPK
			\item findAllByComision: Devuelve un listado de instancias de la entidad filtrando por comision
		\end{itemize}
	}
	\crccolab{
		Comision\\
		ComisionUsuario\\
		ComisionUsuarioPK\\
		ComisionUsuarioRepo\\
	}
\end{crccard}

\begin{crccard}
	\crcclass{DeclaracionService}
	\crcresp{
		\begin{itemize}
			\item la funci�n de esta clase es ser intermediaria con la clase repositorio de la base de datos modificando los m�todos de gesti�n de datos preestablecidos
			\item save: Salva una instancia de la entidad
			\item findAll: Devuelve un listado de instancias de la entidad
			\item deleteAll: Elimina un listado de instancias de la entidad
			\item findByDeclaracionPK: Devuelve una instancia de la entidad filtrando por declaracionPK
			\item existsByDeclaracionPK: Confirma si existe una instancia de la entidad filtrando por declaracionPK
			\item deleteByDeclaracionPK: Elimina una instancia de la entidad filtrando por declaracionPK
			\item findAllByAbierta: Devuelve un listado de instancias de la entidad filtrando por abierta
		\end{itemize}
	}
	\crccolab{
		Declaracion\\
		DeclaracionPK\\
		DeclaracionRepo\\
	}
\end{crccard}

\begin{crccard}
	\crcclass{DenunciaService}
	\crcresp{
		\begin{itemize}
			\item la funci�n de esta clase es ser intermediaria con la clase repositorio de la base de datos modificando los m�todos de gesti�n de datos preestablecidos
			\item save: Salva una instancia de la entidad
			\item findAll: Devuelve un listado de instancias de la entidad
			\item delete: Elimina una instancia de la entidad
			\item findById: Devuelve una instancia de la entidad filtrando por id
			\item deleteAll: Elimina un listado de instancias de la entidad
			\item existsById: Confirma si existe una instancia de la entidad por el id
			\item deleteById: Elimina una instancia de la entidad filtrando por id
			\item findAllByDenunciante: Devuelve un listado de instancias de la entidad filtrando por denunciante
		\end{itemize}
	}
	\crccolab{
		Denuncia\\
		DenunciaRepo\\
	}
\end{crccard}

\begin{crccard}
	\crcclass{DenunciaUsuarioService}
	\crcresp{
		\begin{itemize}
			\item la funci�n de esta clase es ser intermediaria con la clase repositorio de la base de datos modificando los m�todos de gesti�n de datos preestablecidos
			\item save: Salva una instancia de la entidad
			\item findAllByDenuncia: Devuelve un listado de instancias de la entidad filtrando por denuncia
			\item deleteAll: Elimina un listado de instancias de la entidad
			\item existsByDenunciaUsuarioPK: Confirma si existe una instancia de la entidad filtrando por DenunciaUsuarioPK
		\end{itemize}
	}
	\crccolab{
		Denuncia\\
		DenunciaUsuario\\
		DenunciaUsuarioPK\\
		DenunciaUsuarioRepo\\
	}
\end{crccard}

\begin{crccard}
	\crcclass{ExpedienteService}
	\crcresp{
		\begin{itemize}
			\item la funci�n de esta clase es ser intermediaria con la clase repositorio de la base de datos modificando los m�todos de gesti�n de datos preestablecidos
			\item save: Salva una instancia de la entidad
			\item findAll: Devuelve un listado de instancias de la entidad
			\item findById: Devuelve una instancia de la entidad filtrando por id
		\end{itemize}
	}
	\crccolab{
		Expediente\\
		ExpedientePK\\
		ExpedienteRepo\\
	}
\end{crccard}

\begin{crccard}
	\crcclass{PermisoService}
	\crcresp{
		\begin{itemize}
			\item la funci�n de esta clase es ser intermediaria con la clase repositorio de la base de datos modificando los m�todos de gesti�n de datos preestablecidos
			\item saveAll: Salva un listado de instancias de la entidad
			\item findAll: Devuelve un listado de instancias de la entidad
			\item existsByPermiso: Confirma si existe una instancia de la entidad filtrando por permiso
			\item findByPermiso: Devuelve una instancia de la entidad filtrando por permiso
		\end{itemize}
	}
	\crccolab{
		Permiso\\
		PermisoRepo\\
	}
\end{crccard}

\begin{crccard}
	\crcclass{ResolucionService}
	\crcresp{
		\begin{itemize}
			\item la funci�n de esta clase es ser intermediaria con la clase repositorio de la base de datos modificando los m�todos de gesti�n de datos preestablecidos
			\item save: Salva una instancia de la entidad
			\item findAll: Devuelve un listado de instancias de la entidad
			\item findById: Devuelve una instancia de la entidad filtrando por id
			\item deleteAll: Elimina un listado de instancias de la entidad
			\item existsById: Confirma si existe una instancia de la entidad por el id
			\item deleteById: Elimina una instancia de la entidad filtrando por id
		\end{itemize}
	}
	\crccolab{
		PermisoResolucion\\
		ResolucionRepo\\
	}
\end{crccard}

\begin{crccard}
	\crcclass{RolService}
	\crcresp{
		\begin{itemize}
			\item la funci�n de esta clase es ser intermediaria con la clase repositorio de la base de datos modificando los m�todos de gesti�n de datos preestablecidos
			\item save: Salva una instancia de la entidad
			\item saveAll: Salva un listado de instancias de la entidad
			\item findAll: Devuelve un listado de instancias de la entidad
			\item deleteAll: Elimina un listado de instancias de la entidad
			\item findByRol: Devuelve una instancia de la entidad filtrando por rol
			\item existsByRol: Confirma si existe una instancia de la entidad filtrando por rol
			\item deleteByRol: Elimina una instancia de la entidad filtrando por rol
		\end{itemize}
	}
	\crccolab{
		Rol\\
		RolRepo\\
	}
\end{crccard}

\begin{crccard}
	\crcclass{UserDetailsServiceImpl}
	\crcresp{
		\begin{itemize}
			\item loadUserByUsername: Devuelve un objeto compatible con la entidad usuario para realizar la autenticacion
		\end{itemize}
	}
	\crccolab{
		Permiso\\
		Rol\\
		Usuario\\
		UsuarioPrincipal\\
	}
\end{crccard}

\begin{crccard}
	\crcclass{UsuarioService}
	\crcresp{
		\begin{itemize}
			\item la funci�n de esta clase es ser intermediaria con la clase repositorio de la base de datos modificando los m�todos de gesti�n de datos preestablecidos
			\item findAll: Devuelve un listado de instancias de la entidad
			\item deleteAll: Elimina un listado de instancias de la entidad
			\item save: Salva una instancia de la entidad
			\item existsByUsuario: Confirma si existe una instancia de la entidad filtrando por usuario
			\item findByUsuario: Devuelve una instancia de la entidad filtrando por usuario
			\item findAllByCargo: Devuelve un listado de instancias de la entidad filtrando por cargo
		\end{itemize}
	}
	\crccolab{
		Permiso\\
		Rol\\
		Usuario\\
		UsuarioPrincipal\\
	}
\end{crccard}
\end{addendum}
\end{document}
