\chapter{Propuesta de soluci�n}\label{c:chapter2}

\section*{Introducci�n del cap�tulo}
 El presente cap�tulo aborda las particularidades del sistema de gesti�n a desarrollar. Para registrar las principales caracter�sticas se hace uso de los \ac{rf} y \ac{rnf}. Estos describen las funcionalidades y atributos de calidad que debe poseer el software. En el cap�tulo \ref{c:chapter1}, se seleccion� la metodolog�a \ac{xp} como gu�a para el desarrollo del software; por lo tanto, se utilizan las \ac{hu} como herramienta para una descripci�n detallada de los \ac{rf} y la confecci�n del plan de iteraciones. Mediante el uso de este �ltimo, se proceder� a la estimaci�n del tiempo requerido para la culminaci�n del desarrollo del sistema y, con el uso de patrones de dise�o, se facilitar� la posterior descripci�n de las tarjetas \ac{crc}.
 
\section{Requisitos}\label{s:req}

\subsection{Requisitos funcionales}
En ingenier�a de software, los \ac{rf} definen un sistema o sus componentes; describen la funci�n que un software debe realizar, ya sean c�lculos, manipulaci�n de datos, procesos de negocios, entre otros.
Ayudan adem�s a capturar los comportamientos planificados para un sistema, este comportamiento puede ser expresado como una funci�n, servicio o tarea que un software debe realizar \citep{pressman2005software}. A continuaci�n, se exponen los diferentes \ac{rf} planteados por el usuario:
\begin{RF}
    \item Autenticar Usuario
    \item Asignar rol a usuario
    \\
    \item Listar denuncias %w
    \item Crear denuncia
    \item Modificar denuncia
    \item Eliminar denuncia
    \item Buscar denuncia
    %\item Exportar tabla de denuncias
    \item Exportar denuncia
    %\item Imprimir listado de denuncias
    \\
    \item Listar resoluciones decanales %w
    \item Crear resoluci�n decanal %w
    \item Modificar resoluci�n decanal
    \item Eliminar resoluci�n decanal %w
    \item Exportar resoluci�n decanal
    \\
    \item Listar comisiones %w
    \item Crear comisi�n %w
    \item Modificar comisi�n
    \item Eliminar comisi�n %w
    \item Buscar comisi�n %w
    \\
    \item Listar declaraciones
    \item Crear declaraci�n
    \item Modificar declaraci�n
    \item Eliminar declaraci�n
    \item Buscar declaraci�n
    % \item Exportar declaraci�n
    %\item Exportar tabla de declaraciones
    \\
    \item Crear caso disciplinario
    \item Listar casos disciplinarios
    \item Modificar caso disciplinario
    \item Buscar caso disciplinario
    \\
    \item Exportar resoluci�n de caso
    \item Exportar conclusi�n de la comisi�n
    \\
    \item Listar roles
    \item Crear rol
    \item Modificar rol
    \item Eliminar rol
    \item Buscar rol
    %\item Exportar tabla de usuarios
\end{RF}

\subsection{Requisitos no funcionales}
Los \ac{rnf}, como su nombre sugiere, son aquellos requerimientos que no se refieren directamente a las funciones espec�ficas que proporciona el sistema, sino a las propiedades emergentes de este como la fiabilidad, el tiempo de respuesta y la capacidad de almacenamiento. De forma alternativa definen las restricciones del sistema \citep{sommerville2011software}.\\
Los requisitos no funcionales de la aplicaci�n a desarrollar son:
\\
\noindent Usabilidad:
\begin{RNF}
    \item La \ac{gui} de la aplicaci�n cliente debe ser intuitiva y f�cil de usar por cualquier persona con edad laboral y con conocimientos b�sicos de trabajo con software, de acuerdo a est�ndares modernos de dise�o de \ac{gui}.
\end{RNF}
Hardware y software:
\begin{RNF}[resume]
    \item El software debe poder funcionar en un servidor con 4GB de \ac{ram} y 10GB de almacenamiento libre.
    \item El cliente del sistema debe ser compatible con las �ltimas versiones estables de los principales navegadores web: Google Chrome, navegadores basados en \gls{chromium}, Mozilla Firefox y Safari.
    %TODO Seguridad?  \item El usuario debe poder acceder a las funcionalidades del sistema s�lo despu�s de estar autenticado.
    % \item El usuario debe poder acceder solamente a las funcionalidades permitidas de acuerdo a su rol en el sistema.
\end{RNF}
Dise�o e implementaci�n:
\begin{RNF}[resume]
    \item El sistema debe ser accesible a trav�s de una aplicaci�n web cliente.
    \item Se deben usar lenguajes de programaci�n y herramientas com amplio soporte y estabilidad.
\end{RNF}

\section{Historias de usuario}
\label{s:hu}
Las HU ser�n representadas mediante tablas divididas por las siguientes secciones:
\begin{description}

\item N�mero: n�mero de la historia de usuario incremental en el tiempo;
\item Nombre de historia de usuario: el nombre de la historia de usuario ser�a para identificarlas mejor
entre los desarrolladores y el cliente;
\item[Usuario] El usuario que est� involucrado en el desarrollo de la HU;
\item[Iteraci�n asignada] N�mero de la iteraci�n;
\item[Prioridad en negocio] 
\begin{itemize}
\item Las historias de usuarios que son de funcionalidades imprescindibles en el desarrollo del sistema tienen prioridad alta;
\item Las historias de usuarios que son de funcionalidades que debe de tener el sistema, pero que
no son necesarias para su funcionamiento, tienen prioridad media;
\item Las historias de usuarios que son de funcionalidades auxiliares y que son independientes del
sistema, tienen prioridad baja.
\end{itemize}
\item Riesgo en desarrollo:
\begin{itemize}
\item Las historias de usuarios que, en caso de tener alg�n error de implementaci�n, puedan afectar
la disponibilidad del sistema, tienen riesgo de desarrollo alto;
\item Las historias de usuarios que puedan presentar errores y retrasan la entrega de la versi�n,
tienen riesgo de desarrollo medio;
\item Las historias de usuario que puedan presentar errores, pero estos son tratados con facilidad y
no afectan en desarrollo del proyecto, tienen riesgo de desarrollo bajo.
\end{itemize}
\item[Puntos estimados] tiempo estimado que se demorar� el desarrollo de la HU;
\item[Descripci�n] breve descripci�n de la HU;
\item[Observaciones] se�alamiento o advertencia del sistema;
\item[Prototipo de interfaz] prototipo de interfaz si aplica.
\end{description} \citep{Joskowicz2008}


Los t�tulos de las HU generadas son:
\begin{enumerate}[label=HU \arabic*:]
\item  Autenticar usuario.
\item  Gestionar usuario.
\item  Mostrar reportes al asesor.
\item  Gestionar denuncias.
\item  Mostrar clasificaci�n de las faltas.
\item  Gestionar comisiones disciplinarias.
\item  Gestionar expediente disciplinario.
\item  Gestionar dict�menes.
\item  Gestionar pr�rroga.
\item  Gestionar circunstancias modificativas.
\item  Exportar a PDF.
\end{enumerate}
A continuaci�n se presentan las historias de usuarios identificadas en la investigaci�n
\begin{userstory}
	\storyname{Crear denuncia}
	\storyuser{\userY}
	\storyiter{1}
	\storypriority{ High }
	\storyrisk{ Low }
	\storypoints{0.8}
	\storyprogrammer{\authorA }
	\storydescription{}
	\storyobservation{\lipsum[1]}
\end{userstory}

\section{Estimaci�n de esfuerzo por historia de usuario}
\begin {effortestimation}
\addentry[1]{ Listar denuncias }{0.2}
\addentry[1]{ Crear denuncia }{0.4}
\addentry[1]{ Modificar denuncia }{0.4}

\addentry[1]{ Listar comisiones}{0.2}
\addentry[1]{ Crear comisi�n }{0.4}
\addentry[1]{ Modificar comisi�n }{0.4}

\addentry[1]{ Listar resoluciones }{0.2}
\addentry[1]{ Crear resoluci�n }{0.4}
\addentry[1]{ Modificar resoluci�n }{0.4}

\addentry[2]{ Modificar permisos de usuario }{0.2}

\addentry[3]{ Exportar resoluci�n }{0.6}
\addentry[3]{ Exportar expediente }{0.6}
\end{effortestimation}
%\geniterationplan