\chapter{Propuesta de soluci�n}\label{c:chapter2}

\section*{Introducci�n del cap�tulo}
El presente cap�tulo aborda las particularidades del sistema de gesti�n a desarrollar. Para registrar las principales caracter�sticas se hace uso de los \ac{rf} y \ac{rnf}. Estos describen las funcionalidades y atributos de calidad que debe poseer el software. En el cap�tulo \ref{c:chapter1}, se seleccion� la metodolog�a \ac{xp} como gu�a para el desarrollo del software; por lo tanto, se utilizan las \ac{hu} como herramienta para una descripci�n detallada de los \ac{rf} y la confecci�n del plan de iteraciones. Mediante el uso de este �ltimo, se proceder� a la estimaci�n del tiempo requerido para la culminaci�n del desarrollo del sistema y, con el uso de patrones de dise�o, se facilitar� la posterior descripci�n de las tarjetas \ac{crc}.

\subsection{Descripci�n de la propuesta de soluci�n}


\section{Requisitos}\label{s:req}

\subsection{Requisitos funcionales}
En ingenier�a de software, los \ac{rf} definen un sistema o sus componentes; describen la funci�n que un software debe realizar, ya sean c�lculos, manipulaci�n de datos, procesos de negocios, entre otros.
Ayudan adem�s a capturar los comportamientos planificados para un sistema, este comportamiento puede ser expresado como una funci�n, servicio o tarea que un software debe realizar \citep{pressman2005software}. A continuaci�n, se exponen los diferentes \ac{rf} planteados por el usuario:
\begin{RF}
    \item Autenticar Usuario
    \item Asignar rol a usuario
    \\
    \item Listar denuncias %w
    \item Crear denuncia
    \item Modificar denuncia
    \item Eliminar denuncia
    \item Buscar denuncia
    %\item Exportar tabla de denuncias
    \item Exportar denuncia
    %\item Imprimir listado de denuncias
    \\
    \item Listar resoluciones decanales %w
    \item Crear resoluci�n decanal %w
    \item Modificar resoluci�n decanal
    \item Eliminar resoluci�n decanal %w
    \item Exportar resoluci�n decanal
    \\
    \item Listar comisiones %w
    \item Crear comisi�n %w
    \item Modificar comisi�n
    \item Eliminar comisi�n %w
    \item Buscar comisi�n %w
    \\
    \item Listar declaraciones
    \item Crear declaraci�n
    \item Modificar declaraci�n
    \item Eliminar declaraci�n
    \item Buscar declaraci�n
    % \item Exportar declaraci�n
    %\item Exportar tabla de declaraciones
    \\
    \item Crear caso disciplinario
    \item Listar casos disciplinarios
    \item Modificar caso disciplinario
    \item Buscar caso disciplinario
    \\
    \item Exportar resoluci�n de caso
    \item Exportar conclusi�n de la comisi�n
    \\
    \item Listar roles
    \item Crear rol
    \item Modificar rol
    \item Eliminar rol
    \item Buscar rol
    %\item Exportar tabla de usuarios
\end{RF}

\subsection{Requisitos no funcionales}
Los \ac{rnf}, como su nombre sugiere, son aquellos requerimientos que no se refieren directamente a las funciones espec�ficas que proporciona el sistema, sino a las propiedades emergentes de este como la fiabilidad, el tiempo de respuesta y la capacidad de almacenamiento. De forma alternativa definen las restricciones del sistema \citep{sommerville2011software}.\\
Los requisitos no funcionales de la aplicaci�n a desarrollar son:
\\
\noindent Usabilidad:
\begin{RNF}
    \item La \ac{gui} de la aplicaci�n cliente debe ser intuitiva y f�cil de usar por cualquier persona con edad laboral y con conocimientos b�sicos de trabajo con software, de acuerdo a est�ndares modernos de dise�o de \ac{gui}.
\end{RNF}
Hardware y software:
\begin{RNF}[resume]
    \item El software debe poder funcionar en un servidor con 4GB de \ac{ram} y 10GB de almacenamiento libre.
    \item El cliente del sistema debe ser compatible con las �ltimas versiones estables de los principales navegadores web: Google Chrome, navegadores basados en \gls{chromium}, Mozilla Firefox y Safari.
    %TODO Seguridad?  \item El usuario debe poder acceder a las funcionalidades del sistema s�lo despu�s de estar autenticado.
    % \item El usuario debe poder acceder solamente a las funcionalidades permitidas de acuerdo a su rol en el sistema.
\end{RNF}
Dise�o e implementaci�n:
\begin{RNF}[resume]
    \item El sistema debe ser accesible a trav�s de una aplicaci�n web cliente.
    \item Se deben usar lenguajes de programaci�n y herramientas com amplio soporte y estabilidad.
\end{RNF}
				% Requisitos (En XP no se trabaja con listas de requisitos, sino con HU)
\section{Historias de usuario}
\label{s:hu}
Las HU ser�n representadas mediante tablas divididas por las siguientes secciones:
\begin{description}

\item N�mero: n�mero de la historia de usuario incremental en el tiempo;
\item Nombre de historia de usuario: el nombre de la historia de usuario ser�a para identificarlas mejor
entre los desarrolladores y el cliente;
\item[Usuario] El usuario que est� involucrado en el desarrollo de la HU;
\item[Iteraci�n asignada] N�mero de la iteraci�n;
\item[Prioridad en negocio] 
\begin{itemize}
\item Las historias de usuarios que son de funcionalidades imprescindibles en el desarrollo del sistema tienen prioridad alta;
\item Las historias de usuarios que son de funcionalidades que debe de tener el sistema, pero que
no son necesarias para su funcionamiento, tienen prioridad media;
\item Las historias de usuarios que son de funcionalidades auxiliares y que son independientes del
sistema, tienen prioridad baja.
\end{itemize}
\item Riesgo en desarrollo:
\begin{itemize}
\item Las historias de usuarios que, en caso de tener alg�n error de implementaci�n, puedan afectar
la disponibilidad del sistema, tienen riesgo de desarrollo alto;
\item Las historias de usuarios que puedan presentar errores y retrasan la entrega de la versi�n,
tienen riesgo de desarrollo medio;
\item Las historias de usuario que puedan presentar errores, pero estos son tratados con facilidad y
no afectan en desarrollo del proyecto, tienen riesgo de desarrollo bajo.
\end{itemize}
\item[Puntos estimados] tiempo estimado que se demorar� el desarrollo de la HU;
\item[Descripci�n] breve descripci�n de la HU;
\item[Observaciones] se�alamiento o advertencia del sistema;
\item[Prototipo de interfaz] prototipo de interfaz si aplica.
\end{description} \citep{Joskowicz2008}


Los t�tulos de las HU generadas son:
\begin{enumerate}[label=HU \arabic*:]
\item  Autenticar usuario.
\item  Gestionar usuario.
\item  Mostrar reportes al asesor.
\item  Gestionar denuncias.
\item  Mostrar clasificaci�n de las faltas.
\item  Gestionar comisiones disciplinarias.
\item  Gestionar expediente disciplinario.
\item  Gestionar dict�menes.
\item  Gestionar pr�rroga.
\item  Gestionar circunstancias modificativas.
\item  Exportar a PDF.
\end{enumerate}
A continuaci�n se presentan las historias de usuarios identificadas en la investigaci�n
\begin{userstory}
	\storyname{Crear denuncia}
	\storyuser{\userY}
	\storyiter{1}
	\storypriority{ High }
	\storyrisk{ Low }
	\storypoints{0.8}
	\storyprogrammer{\authorA }
	\storydescription{}
	\storyobservation{\lipsum[1]}
\end{userstory}
					% Historias de usuario
\section{Estimaci�n de esfuerzo por historia de usuario}

\begin {effortestimation}
\addentry[1]{ Listar denuncias }{\hpr}
\addentry[1]{ Crear denuncia }{\hpc}
\addentry[1]{ Modificar denuncia }{\hpu}
\addentry[1]{ Eliminar denuncia }{\hpd}
\addentry[1]{ Buscar denuncia }{\hpb}

\addentry[1]{ Listar comisiones}{\hpr}
\addentry[1]{ Crear comisi�n }{\hpc}
\addentry[1]{ Modificar comisi�n }{\hpc}
\addentry[1]{ Eliminar comisi�n }{\hpc}
\addentry[1]{ Buscar comisi�n }{\hpc}

\addentry[1]{ Listar resoluciones }{\hpr}
\addentry[1]{ Crear resoluci�n }{\hpc}
\addentry[1]{ Modificar resoluci�n }{\hpu}
\addentry[1]{ Eliminar resoluci�n }{\hpd}
\addentry[1]{ Buscar resoluci�n }{\hpb}

\addentry[2]{ Autenticar usuario }{\hpa}

\addentry[2]{ Modificar roles de usuario }{\hpu}
\addentry[2]{ Listar rol}{\hpr}
\addentry[2]{ Crear rol}{\hpc}
\addentry[2]{ Modificar rol}{\hpu}
\addentry[2]{ Eliminar rol}{\hpd}
\addentry[2]{ Buscar rol}{\hpb}

\addentry[3]{ Exportar resoluci�n de caso }{\hpe}
\addentry[3]{ Exportar conclusi�n de la comisi�n }{\hpe}
\addentry[3]{ Exportar tabla de denuncias }{\hpr}
\end{effortestimation}

\pagebreak 
	% Estimaci�n de esfuerzo por historia de usuario

\section{Plan de duraci�n de las iteraciones}
\geniterationplan
% Separar generaci�n del plan de iteraciones del siguiente input porque si no no se genera el siguiente input

\section{Tareas}
La metodolog�a de software XP plantea que la implementaci�n de un software se hace iterativamente.
Durante cada iteraci�n se desarrollan un conjunto de \ac{hu} definidas por el cliente y descritas por el equipo de desarrollo. En esta fase de implementaci�n las \ac{hu} se dividen en tareas, las cuales son asignadas a los programadores para ser implementadas durante la iteraci�n correspondiente \citep{Joskowicz2008}.\\
A continuaci�n, se muestran algunas de las tareas de ingenier�a a realizar:

\subsection{Iteraci�n I}
Para su desarrollo durante la primera iteraci�n se acuerda la selecci�n de las \ac{hu} con mayor prioridad,respetando la opini�n del cliente, cuya suma del tiempo total estimado de desarrollo no exceda el tiempo acordado para las iteraciones ( 2 semanas ). A continuaci�n los nombres de algunas de las \ac{hu} seleccionadas para la iteraci�n junto a las tareas que, una vez implementadas satisfactoriamente, dan paso a la realizaci�n de las pruebas de aceptaci�n:

\begin{itemize}
\item \textbf{HU4:} Realizar denuncia
\end{itemize}
% \begin{tasks}

% 	\item Implementar: Crear denuncia.
% 	\item Implementar: Modificar denuncia.
% 	\item Implementar: Eliminar denuncia.
% 	\item Implementar: Consultar denuncia.
% 	\item Implementar: Listar denuncias.
% 	\item Implementar: Buscar denuncia.
% 	\item Implementar: Exportar denuncia.
% 	\item Implementar: Autenticar usuario.
% 	\item Implementar: Asignar rol a usuario.
% 	\item Implementar: Crear rol
% 	\item Implementar: Modificar rol.
% 	\item Implementar: Eliminar rol.
% 	\item Implementar: Listar roles.
% 	\item Implementar: Crear resoluci�n decanal
% 	\item Implementar: Listar resoluciones decanales.
% 	\item Implementar: Consultar resoluci�n decanal.
% 	\item Implementar: Exportar resoluci�n decanal.
	
% \end{tasks}


\subsection{Iteraci�n II}
Para su desarrollo durante la segunda iteraci�n se acuerda la selecci�n de las \ac{hu} con prioridad de segundo orden,respetando la opini�n del cliente, cuyo cantidad no exceda la el n�mero de \ac{hu} que se pudieron desarrollar y probar exitosamente en la iteraci�n anterior, lo que se conoce como velocidad de desarrollo. Luego se verifica que las cantidad de tareas total asociadas a las \ac{hu} no excede tampoco la velocidad de desarrollo de la iteraci�n anterior. Y de esta manera se procede con todas las iteraciones siguientes. A continuaci�n los nombres de algunas de las \ac{hu} seleccionadas para la iteraci�n junto a las tareas que, una vez implementadas satisfactoriamente, dan paso a la realizaci�n de las pruebas de aceptaci�n:

\begin{itemize}
	
\item \textbf{HU1:} Implemmentar: Autenticar usuario
\item \textbf{HU2:} Listar usuarios
\item \textbf{HU3:} Asignar rol a usuario
\end{itemize}

\begin{engineeringtask}[t:engtask1] % label in brackets
    \engtaskuserstory{4}
    \engtaskname{Autenticar usuario}
    \engtasktype{Desarrollo}
    \engtaskpointestimation{1}
    \engtaskstartdate{6}{5}{2014} % day, month, year
    \engtaskenddate{24}{10}{2014}
    \engtaskdescription{Implementar la autenticaci�n basada en jwt.}
    \engtaskprogrammer{John Doe}
 \end{engineeringtask}
% \begin{developmenttask}[t:devtask1] % label in brackets
%     \devtaskuserstory{4}
%     \devtaskname{Implementar: Autenticar usuario}
%     \devtasktype{Desarrollo}
%     \devtaskpointestimation{1}
%     \devtaskstartdate{1}{10}{2022} % day, month, year
%     \devtaskenddate{2}{10}{2022}
%     \devtaskdescription{La autenticaci�n debe funcionar usando el servicio de autenticaci�n de la universidad.}
%     \devtaskprogrammer{\authorA}
%  \end{developmenttask}
\section{Tarjetas CRC}\label{crc}

\begin{crccard}
   \crcclass{DenunciaController}
   \crcresp{
      \begin{itemize}
         \item Crear una denuncia
         \item Mostrar la denuncia requerida
         \item Eliminar una denuncia
      \end{itemize}
   }
   \crccolab{
      UsuarioService\\
      DenunciaService\\
      Convertidor\\
      SesionDetails\\
      ValidatorDenuncia\\
      Mensaje\\
      JsonCrearDenuncia\\
      JsonModificarDenuncia\\
      JsonBorrarDenuncia\\
   }
\end{crccard}
\begin{crccard}
\crcclass{ComisionController}
    \crcresp{
       \begin{itemize}
          \item Mostrar las comisiones existentes
          \item Modificar una comisi�n
          \item Eliminar una comisi�n
       \end{itemize}
    }
   \crccolab{
      ResolucionService\\
      ComisionService\\
      UsuarioService\\
     RolService\\
     ComisionUsuarioService\\
    Mensaje\\
    ValidatorComision\\
   ComisionUsuario\\
   UsuarioRol\\
JsonModificarComision\\
Comision\\
ComisionUsuarioPK\\
JsonBorrarComision\\
   }
 \end{crccard}
 
\begin{crccard}
   \crcclass{CasoController}
    \crcresp{
       \begin{itemize}
          \item Crear el caso que asocia una comisi�n con una denuncia
          \item Crear el dictamen que emite dicho caso al finalizar
          \item Mostrar el listado de casos
          \item Editar un caso requerido
         \item Eliminar un caso no deseado
       \end{itemize}
    }
   \crccolab{
      GestionarFicheros\\
      DenunciaService\\
      ComisionService\\
     CasoService\\
     Convertidor\\
    Mensaje\\
    SesionDetails\\
   ValidatorCaso\\
   JsonCrearCaso\\
Caso\\
Denuncia\\
JsonCrearDictamen\\
JsonModificarCaso\\
CasoPK\\
JsonBorrarCasos\\
   }
 \end{crccard}    % Tarjetas CRC
\begin{figure}[h]
	\centering
	\includegraphics[width=1\textwidth]{images/entities.jpeg}
	\caption{Diagrama entidad-relaci�n a partir del cual se genera la base de datos relacional de CDIS.}
	\label{fig:entities}
\end{figure}
\section{Patrones de dise�o}
Un patr�n de dise�o es una abstracci�n de una soluci�n en un nivel alto. Los patrones solucionan problemas que existen en muchos niveles de abstracci�n. Hay patrones que abarcan las distintas etapas del desarrollo \citep{giraldo2011disenosoft}.

\subsection{Patrones GRASP} Lo esencial de un dise�o de objetos lo constituye el dise�o de las interacciones de objetos y la asignaci�n de responsabilidades. Las decisiones que se tomen pueden influir profundamente en la extensibilidad, claridad y mantenimiento del sistema de software de objetos, adem�s en el grado y calidad de los componentes reutilizables, por esta raz�n, durante el dise�o se deben realizar los casos de usos con objetos basado en los \ac{grasp} \citep{giraldo2011disenosoft}.
Algunos ejemplos de patrones \ac{grasp} utilizados en la implementaci�n de la soluci�n son los siguientes.

\paragraph{Creador:}

Este patr�n implica que un objeto debe responsabilizarse de crear otros:
\begin{itemize}
    \item Si contiene o agrega varios objetos del tipo de los creados.
    \item Si se encarga de registrar objetos del tipo de los creados.
    \item Si utiliza mucho los objetos creados.
    \item O si contiene los datos para crear los del tipo creados.
\end{itemize}

Ejemplo de uso de patr�n creador en la implementaci�n de la propuesta de soluci�n:
\begin{figure}[htp]
    \centering
    \includegraphics[width=0.6\textwidth]{images/patterns/creator.jpg}
    \caption{Demostraci�n de uso del patr�n controlador}
    \label{fig:creator}
\end{figure}

En la imagen anterior se demuestra el uso del patr�n creador pues la
clase ``CasoService'' hace bastante uso por medio de los m�todos que
dispone de los objetos existentes en la base de datos, hace registros de
los mismos con frecuencia y posee los datos necesarios para crear dichos
objetos.

\paragraph{Controlador \emph{(controller)}:}

Es un patr�n por el cual definimos objetos llamados controladores que
independizan las interfaces con las acciones que haya que hacer.

Estos controladores tienen que ser lo m�s peque�os posible y numerosos,
aislando todo lo posible las capas internas de los programas de las
interfaces.

Ejemplo de uso del patr�n controlador en la implementaci�n de la propuesta de soluci�n: \\
\begin{figure}[htp]
    \centering
    \includegraphics[width=0.6\textwidth]{images/patterns/controller.jpg}
    \caption{Demostraci�n de uso del patr�n controlador}
    \label{fig:controller}
\end{figure}

En la imagen anterior se demuestra el uso del patr�n controlador pues la
clase ``CasoController'', entre otras se�aladas, independizan las
interfaces con acciones que se deben realizar, adem�s de que separan, en la
medida de lo posible, las capas internas de la programaci�n de la
aplicaci�n de su interfaz.




\subsection{Patrones GoF}
En el a�o 1994, apareci� el libro ``Design Patterns: Elements of Reusable Object Oriented Sofware'' escrito por los ahora famosos \emph{Gang of Four} (Pandilla de los cuatro) integrada por Erich Gamma, Richard Helm, Ralph Johnson y John Vlissides. Estos recopilaron y documentaron 23 patrones de dise�o aplicados usualmente por expertos dise�adores de software orientado a objetos. Desde luego ellos no son los inventores ni los �nicos involucrados, pero luego de la publicaci�n de ese libro empez� a difundirse con m�s fuerza la idea de patrones de dise�o. Se distinguen tres tipos de patrones GoF: patrones de comportamiento, patrones creacionales y patrones estructurales \citep{giraldo2011disenosoft}.

\paragraph{�nico \emph{(singleton)}:}
Garantiza que una clase s�lo tenga una instancia, y proporciona un punto de acceso global a ella.\\
\paragraph{ Decorador \emph{(decorator)}:} A�ade funcionalidad a una clase din�micamente. De esta manera, instancias aisladas pueden poseer funcionalidades extra s�lo en tiempo de ejecuci�n.
\paragraph{ Fachada \emph{(facade)}:} Provee de una interfaz unificada simple para acceder a una interfaz o grupo de interfaces de un subsistema.

Ejemplo de uso del patr�n �nico, decorador y fachada en la implementaci�n de la soluci�n:\\

\begin{figure}[h]
    \centering
    \includegraphics[width=0.6\textwidth]{images/patterns/autowired.png}
    \caption{Demostraci�n de uso del patr�n �nico}
    \label{fig:singleton}
\end{figure}
En la imagen se muestra como se inyecta una instancia �nica de varios objetos que se podr�n usar en una clase controladora para tenerlos como punto de acceso global a ella.
\subsubsection{Patrones creacionales:} Tratan con las formas de crear instancias de objetos. El objetivo de estos patrones es de abstraer el proceso de instanciaci�n y ocultar los detalles de c�mo los objetos son creados o inicializados.
\paragraph{Representante \emph{(proxy)}:} Mantiene un representante de un objeto.
\begin{figure}[htp]
    \centering
    \includegraphics[width=0.8\textwidth]{images/patterns/proxy.png}
    \caption{Uso del patr�n Proxy en la implementaci�n de la soluci�n}
\end{figure}
\subsubsection*{Patrones estructurales:} Describen c�mo clases y objetos pueden ser combinados para formar grandes estructuras y proporcionar nuevas funcionalidades. Estos objetos adicionados pueden ser incluso objetos simples u objetos compuestos.
Proporciona un modo de acceder secuencialmente a los elementos de un objeto agregado sin exponer su representaci�n interna.
Representa y externaliza el estado interno de un objeto sin violar la encapsulaci�n, de forma que �ste puede volver a dicho estado m�s tarde.
\paragraph{ Observador \emph{(observer)}:}  Define una dependencia de uno-a-muchos entre objetos, de forma que cuando un objeto cambie de estado se notifique y actualicen autom�ticamente todos los objetos que dependen de �l.
\begin{figure}[htp]
    \centering
    \includegraphics[width=0.8\textwidth]{images/patterns/watch.png}
    \caption{Uso del patr�n Observador en la implementaci�n de la soluci�n}
\end{figure}
\paragraph{ Estado \emph{(state)}:} Permite que un objeto modifique su comportamiento cada vez que cambia su estado interno. Parecer� que cambia la clase del objeto.

\section{Est�ndares de codificaci�n}
Entendemos como est�ndar de c�digo a un conjunto de convenciones establecidas de ante mano (denominaciones, formatos, etc.) para la escritura de c�digo \citep{vera2019mejores}.

\paragraph{Comentario de c�digo:}
\begin{itemize}
    \item Los comentarios deben ser oraciones completas.
    \item Si un comentario es una frase u oraci�n su primera palabra debe comenzar con may�scula a menos que sea un identificador que comience con min�scula.
\end{itemize}

\paragraph{M�xima longitud de las l�neas:}

\begin{itemize}
    \item Se limitar�n todas las l�neas a un m�ximo de 150 caracteres.
    \item Dentro de par�ntesis, corchetes o llaves se puede utilizar la continuaci�n impl�cita para cortar las l�neas largas.
\end{itemize}

\paragraph{L�neas en blanco:}
\begin{itemize}
    \item Las definiciones de m�todos dentro de una clase deben separarse por una l�nea en blanco.
    \item Se puede utilizar l�neas en blanco escasamente para separar secciones l�gicas.
\end{itemize}

\paragraph{Importaciones:}
\begin{itemize}
    \item Las importaciones deben estar en l�neas separadas.
    \item Las importaciones siempre deben colocarse al comienzo del archivo.
\end{itemize}
\section{Arquitectura de la propuesta de soluci�n}
Para garantizar niveles elevados de calidad de software durante el desarrollo es necesario definir desde el comienzo una arquitectura que describa sus principios fundamentales, garantizando robustez y escalabilidad.
La arquitectura de software define la estructura del sistema, la cual est� constituida por componentes, con funciones espec�ficas, que interact�an entre s� \citep{navarro2018arquitectura}.
\subsection{Patr�n Arquitect�nico}
Para el desarrollo de la propuesta de soluci�n se utiliza el patr�n arquitect�nico \ac{mvc} debido a que este separa la l�gica de negocio de la interfaz de usuario en tres capas diferentes, cada una con funcionalidades bien definidas, reduciendo esfuerzo en la implementaci�n de la aplicaci�n y garantizando una mejor organizaci�n del trabajo.
\subsubsection*{Partes del \ac{mvc}:}
\paragraph{ Modelo:} Contiene una representaci�n de los datos que maneja el sistema, su l�gica de negocio, y sus mecanismos de persistencia.
\paragraph{ Vista:} Tambi�n conocida como interfaz de usuario, compone la informaci�n que se env�a al cliente y los mecanismos de interacci�n con este.
\paragraph{ Controlador:} Act�a como intermediario entre el Modelo y la Vista, gestionando el flujo de informaci�n entre ellos y las transformaciones para adaptar los datos a las necesidades de cada uno.\\
\begin{figure}[htp]
	\centering
	\includegraphics[width=0.5\textwidth]{images/patterns/controller.jpg}
	\caption{Ejemplo de la implmentaci�n de una clase controladora en el c�digo del sistema}
	\label{fig:arch-controller}
\end{figure}

\section*{Conclusiones del cap�tulo}
Al finalizar el presente cap�tulo se arriba a las siguientes conclusiones parciales:

\begin{itemize}
	\item El modelado del proceso de negocio permiti� definir las actividades a automatizar dentro del mismo.
	\item La especificaci�n de los requisitos mediante historias de usuario de conjunto con un listado hizo posible tener una visi�n clara de los objetivos de la implementaci�n y permiti� confeccionar un plan de iteraciones basado en la prioridad de los requisitos definida por el cliente de conjunto con el equipo de desarrollo.
	\item El empleo de patrones y est�ndares permiti� llevar a cabo una implementaci�n estable con buena comunicaci�n dentro del equipo de desarrollo, lo cual tribut� a aumentar la velocidad de desarrollo del equipo calculada para cada iteraci�n.
	\item Se obtuvieron las tarjetas CRC correspondientes a cada una de las clases.
	\item Finalmente, se definieron las tareas de ingenier�a y los est�ndares de codificaci�n que servir�n de gu�a para alcanzar un producto no solo funcional, sino de c�digo legible y escalable.
\end{itemize}