\begin{table}[h]
	\centering
	\begin{tabular}{|l|l|l|}
		\hline
		\theader
		\textbf{ Aspectos } 	& \textbf{Metodolog�as �giles}   & \textbf{ Metodolog�as Pesadas} \\ \hline
		Acercamiento 			& Adaptivo 				  & Predictivo 				\\ \hline
		\stripe
		Medici�n del objetivo   & Valor para el negocio   & Conformaci�n del plan	\\ \hline
		Tama�o del proyecto		& Peque�o 				  & Largo 					\\ \hline
		\stripe
		Tipo de administraci�n  & Descentralizado 		  & Autocr�tico 			\\ \hline
		Perspectiva de cambios  & Adaptable a cambio 	  & Sostenible al cambio 	\\ \hline
		\stripe
		Cultura del equipo      & Liderazgo\textbackslash Colaboraci�n & Comandada\textbackslash Controlada \\ \hline
		Documentaci�n		    & Baja 					  & Pesada 					\\ \hline
		\stripe
		Orientada a 		    & Personas 				  & Procesos 				\\ \hline
		Ciclos de desarrollo    & Numerosos 			  & Limitados 				\\ \hline
		\stripe 
		Dominio de desarrollo   & Impredecibles 		  & Predecibles 			\\ \hline
		Planificaci�n inicial   & M�nima   				  & Comprensiva 			\\ \hline
		\stripe 
		Retorno de la inversi�n & Temprano en el proyecto & Al final del proyecto 	\\ \hline
		Tama�o del equipo 		& Peque�o 				  & Grande 					\\ \hline
		
	\end{tabular}
	\caption{Comparaci�n, Metodolog�as �giles VS Metodolog�as Tradicionales o Pesadas }%~\citep{Awad2005}}
	\label{table:metodologies-comparison}
\end{table}