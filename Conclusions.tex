\conclusions
Tras finalizar la presente investigaci�n se puede concluir que se desarrollaron todas las tareas que tributan al cumplimiento de los objetivos propuestos, resaltando que:
\begin{itemize}
    \item El estudio de la bibliograf�a y an�lisis de los sistemas existentes para la gesti�n del \proceso\ demostr� la necesidad del desarrollo de un nuevo sistema inform�tico para llevar a cabo la gesti�n del mismo en \lafac.
    \item Durante la planificaci�n y el dise�o de la soluci�n se generaron los artefactos propuestos por la metodolog�a de desarrollo seleccionada, lo cual permiti� facilitar la implementaci�n de las funcionalidades definidas.
    \item La implementaci�n de la propuesta de soluci�n contribuye a la gesti�n del \proceso\ en \lafac.
    \item Las pruebas de software realizadas permitieron garantizar un correcto funcionamiento de la herramienta de acuerdo a lo planificado en cada entrega, as� como el cumplimiento de las necesidades y requisitos del cliente.
\end{itemize}
%Se cumple el objetivo general de desarrollar un \cdis luego de haber analizado elementos te�ricos y principales tendencias del desarrollo de sistemas en la actualidad, espec�ficamente en el campo de sistemas de gesti�n de la informaci�n, definido las tecnolog�as y herramientas necesarias para el desarrollo de la propuesta de soluci�n, implementado y evaluado las funcionalidades de la misma.
% Al finalizar la presente investigaci�n se conclye que el proceso de comisi�n disciplinaria que se lleva a cabo en la Facultad 4 de la Universidad de las Ciencias Inform�ticas, el cual se realiza de manera manual puede ejecutarse ahora de manera semi-automatizada a trav�s del Sistema para la Gesti�n del Proceso de Comisi�n Disciplinaria en la Facultad 4 de la Universidad de las Ciencias Inform�ticas. Siendo este, adem�s, una soluci�n escalable y adaptable, lo cual facilitar� la implementaci�n de nuevas funcionalidades.
