\chapter{Fundamentaci�n te�rica}\label{c:chapter1}
\section*{Introducci�n del cap�tulo}
En el presente cap�tulo se engloban aspectos relacionados con el objeto de estudio definido para el problema planteado. El an�lisis de algunas metodolog�as, procedimientos, herramientas existentes para el desarrollo de sistemas web y el an�lisis de aplicaciones hom�logas; permitir� la selecci�n de las tecnolog�as adecuadas para el desarrollo del \cdis\ y contar con un an�lisis de sistemas existentes que realizan funcionalidades similares.

\section{Proceso de comisi�n disciplinaria en la Facultad 4}\label{process}
%TODO
%Descripci�n del proceso de comisi�n disciplinaria en la Facultad 4
En \lafac\ el \proceso\ se rige por los normas establecidas en \elreglamento\ descrito en \laresolucion. El mismo define que para la tramitaci�n de las denuncias por faltas disciplinarias se designar� en cada Facultad no menos de dos Comisiones Disciplinarias al inicio de cada Curso Acad�mico, las que ejercer�n sus funciones hasta que sean designadas las Comisiones del siguiente curso. Cada una estar� integrada por dos miembros del claustro de profesores, uno de los cuales la presidir� y el otro actuar� como Secretario, las que ser�n designadas por el Decano que, en cada oportunidad en que reciba una denuncia de falta disciplinaria deber� coordinar la participaci�n de la \ac{feu} del nivel, que actuar� de vocal.\\

El proceso de comisi�n disciplinaria comienza cuando una persona presenta una denuncia ante el Decano de la Facultad. Este es el encargado de verificar que la misma procede y de asignar una comisi�n disciplinaria para que investigue el caso.\\
Las faltas disciplinarias deben ponerse en conocimiento de los Vicerrectores, Decanos, Vicedecanos, Jefes de Departamento o miembros del personal docente, tan pronto se produzcan o se conozcan por cualquier trabajador o estudiante del centro de educaci�n superior. Cuando las faltas disciplinarias se informen verbalmente se levantar� acta firmada por el o los denunciantes con el funcionario que designe el Decano de la Facultad. Transcurridos 60 d�as h�biles despu�s de denunciada una falta sin que se inicie el proceso, se proceder� al archivo de la misma, sin m�s tr�mites, debiendo depurarse las responsabilidades por tal motivo, si las hubiere.\\

Si la indisciplina fuera cometida por alumnos becarios en la Residencia Estudiantil, se proceder� en la forma siguiente:\\
Cuando la falta sea cometida por un becario en el �rea de la Residencia Estudiantil, el Director de Becas dar� cuenta al Decano de la Facultad a que pertenezca el alumno, dentro de la 24 horas de haber sucedido el hecho, para que el alumno sea juzgado por la Comisi�n Disciplinaria constituida en la Residencia a instancia del Decano de su Facultad, la cual para su mejor informaci�n sobre lo sucedido deber� ser ampliada con un representante de la Direcci�n de Becas y uno de la Federaci�n Estudiantil Universitaria del edificio en que ocurri� el hecho.\\

La Comisi�n Disciplinaria efectuar� las diligencias siguientes:\\
a) ocupar� las pruebas documentales o materiales de la infracci�n, si existieren, y conocer� los resultados de las Asambleas Estudiantiles cuando se hayan realizado;\\
b) tomar� declaraci�n sobre los hechos por escrito y bajo firma a los informantes;\\
c) citar� al estudiante para instruirlo de cargos, oyendo y dejando constancia por escrito de los descargos que formula.\\
d) Practicar� cuantas pruebas testif�cales se propongan.\\
e) oir� el criterio del profesor gu�a sobre el infractor, tom�ndole declaraciones por escrito;\\
f) examinar� los expedientes acad�micos correspondientes;\\
g) practicar� de oficio o a instancia de parte, cualquier otra diligencia de prueba que procediere;\\
h) asesorar� a los estudiantes en cuanto a sus deberes y derechos y sobre el procedimiento a seguir en los casos de faltas disciplinarias.\\

La Comisi�n Disciplinaria conocer� de la opini�n de la Brigada Estudiantil a trav�s de la Federaci�n Estudiantil Universitaria y la Uni�n de j�venes Comunistas, por escrito.\\
Si se proponen pruebas, la Comisi�n Disciplinaria deber� practicarlas dentro del t�rmino establecido para sus actuaciones. Si dicho t�rmino estuviese pr�ximo a vencerse, el Decano o Director de Sede Universitaria Municipal, Unidad Docente o Filial conceder� hasta diez d�as h�biles m�s, dentro de los cuales estar� obligado a practicarlas. El t�rmino para las actuaciones en los expedientes disciplinarios ser� el de 30 d�as, por lo que, en ning�n caso, el c�mputo del t�rmino para la pr�ctica de las actuaciones ser� mayor de 40 d�as h�biles, transcurridos los cuales se dar� por terminada la labor de la comisi�n sin perjuicio de la responsabilidad administrativa en que puedan haber incurrido sus miembros.\\
Terminadas sus actuaciones, la Comisi�n Disciplinaria las elevar� con sus
conclusiones al Decano, el cual, si est� de acuerdo, dictar� la resoluci�n correspondiente \citep{resolucion240-07}.

%%%%%
El proceso de comisi�n disciplinaria comienza cuando una persona de la \uci presenta una denuncia ante el Decano de una Facultad. Este es el encargado de verificar la veracidad de la denuncia y de asignar una comisi�n disciplinaria para que investigue el caso. Esta investigaci�n incluye la recopilaci�n de informaci�n necesaria para aclarar los hechos: declaraciones de implicados y opiniones de los trabajadores que atienden las �reas a las que pertenece cada infractor; y de emitir conclusiones del caso para cada uno. Estas son revisadas por el Decano de la Facultad, el cual, de aceptarlas, emite la resoluci�n del caso y notifica a cada infractor \citep{resolucion240-07}.
\begin{figure}[h]
	\centering
	\includegraphics[width=1\textwidth]{images/process.png}
	\caption{Diagrama de actividades que describe el \proceso\ de \lafac}
	\label{fig:activities}
\end{figure}


\section[Conceptos]{Principales conceptos asociados al proceso de Comisi�n Disciplinaria en la Facultad 4}\label{concepts}

\newcommand{\ectx}{En el contexto del proceso de comisi�n disciplinaria se refiere}
\comment[carlosepc]{Buscar y citar las fuentes faltantes de las explicaciones de los conceptos.}
\paragraph{Denuncia:} Es la acci�n y efecto de denunciar (avisar, noticiar, declarar la irregularidad o ilegalidad de algo, delatar). La denuncia puede realizarse ante las autoridades correspondientes (lo que implica la puesta en marcha de un mecanismo judicial) o de forma p�blica (s�lo con valor testimonial) \citep{rae}. \ectx al documento emitido por un trabajador de la \uci en el que describe una indisciplina cometida por uno o varios estudiantes; con el objetivo de que se ponga en mar
\paragraph{Comisi�n disciplinaria:} Conjunto de personas encargadas por una autoridad de velar por la buena conducta y disciplina \citep{rae}. En el contexto del proceso analizado en este trabajo se refiere al equipo conformado por un jefe y un secretario, ambos profesores, que se encarga de la resoluci�n de un caso disciplinario. Si la indisciplina asociada al caso fue realizada en la residencia, pasan a formar parte de la comisi�n disciplinaria el representante de la residencia, quien es un trabajador de la misma, y un representante del edificio donde vive el estudiante.

\paragraph{Caso disciplinario:} Se crea cuando se aprueba una denuncia y se asigna una comisi�n disciplinaria para su an�lisis.

\paragraph{Expediente:} Instrumento administrativo que recopila la documentaci�n imprescindible que sustenta un acto administrativo \citep{rae}. En el contexto del proceso de comisi�n disciplinaria se refiere a resumen de la documentaci�n generada por un caso disciplinario.

\paragraph{Gesti�n:} Del lat�n gest\u{i}o, el concepto de gesti�n hace referencia a la acci�n y a la consecuencia de administrar o gestionar algo. Al respecto, hay que decir que gestionar es llevar a cabo diligencias que hacen posible la realizaci�n de una operaci�n comercial o de un anhelo cualquiera \citep{definicionde}.

\paragraph{Sistema de gesti�n:}

\paragraph{Servidor web:} Un servidor web o Servidor HTTP es una pieza de software de comunicaciones que intermedia entre el servidor en el que est�n alojados los datos solicitados y el computador del cliente, permitiendo conexiones bidireccionales o unidireccionales, s�ncronas o as�ncronas, con cualquier aplicaci�n del cliente, incluso con los navegadores que traducen un c�digo (renderizable) a una p�gina web determinada. O sea, se trata de programas que median entre el usuario de Internet y el servidor en donde est� la informaci�n que solicita.

\paragraph{Aplicaci�n cliente:} Una aplicaci�n cliente es un paquete de software que funciona sobre el propio sistema operativo de un dispositivo (que puede ser por ejemplo un smartphone, una laptop o un equipo de escritorio). Es decir, que la aplicaci�n se instala y corre �dentro� del computador, o lo que solemos llamar tambi�n �de forma local�. En un computador, estas aplicaciones se instalan en el disco duro, donde guardan toda la informaci�n. 

\paragraph{Aplicaci�n web:}
\section[Sistemas hom�logos]{Sistemas para la gesti�n de procesos de correcci�n y sanciones, utilizados a nivel nacional e internacional}

\subsection{Sistemas en el �mbito internacional}

\subsection{Sistemas en el �mbito nacional}
\begin{itemize}
	\item CDis
	\item SGPCD
	\item CODIS
\end{itemize}

\comment[CarlosE]{Revisar las referencias bibliogr�ficas}

SIGPCD \cite{Awad2005}
	

%	SECTION
\section{Herramientas y tecnolog�as a utilizar}

Para el desarrollo de cualquier aplicaci�n, es necesario utilizar diferentes t�cnicas como: los patrones de dise�o o las metodolog�as de desarrollo de software, adem�s del uso de distintas herramientas como los compiladores o editores de c�digos. Aunque a simple vista parezca que la selecci�n de las tecnolog�as para desarrollar aplicaciones es f�cil, es totalmente lo contrario; para su correcta selecci�n, es necesario ver el problema a resolver desde diferentes �ngulos y posibles situaciones futuras. El presente ep�grafe aborda alguna de las diferentes herramientas que dan soluci�n a la problem�tica planteada.

\subsection{Patrones de dise�o: }

Seg�n el libro Dive Into Design Patterns, los patrones de dise�o son:
\begin{quote}
	``Soluciones t�picas a problemas comunes en el desarrollo de software. Se podr�a decir que son como planos predefinidos que pueden ser adaptados para resolver problemas en el dise�o de la codificaci�n de un programa~\citep{Shevts2019}.''
\end{quote}

Normalmente se confunden con algoritmos, porque ambos conceptos describen soluciones t�picas a problemas conocidos, mientras que un algoritmo describe una serie de pasos a seguir para lograr un objetivo, un patr�n es una descripci�n de alto nivel de la soluci�n, o sea, la codificaci�n de un mismo patr�n puede ser diferente en programas distintos~\citep{Shevts2019}.

Los patrones de dise�o difieren entre ellos debido a su complejidad, el nivel de detalles necesarios y la escala del sistema que se va a implementar. Los patrones de bajo nivel son llamados idiomas y usualmente solo se aplican a un lenguaje de programaci�n. Mientras que los patrones m�s universales y de m�s alto nivel, son llamados patrones arquitect�nicos. Estos �ltimos pueden ser usados en cualquier programa independiente del lenguaje en que sea programado y adem�s pueden ser utilizados para crear la arquitectura completa de un software~\citep{Shevts2019}. 

\subsubsection{Surgimiento de los patrones:}

En un principio, no fueron llamados patrones, ni estaban agrupados, sino que fueron soluciones que se repitieron una y otra vez en el desarrollo de software. Debido a esto, Erich Gamma, Jhon Vlissides, Ralph Johnson y Richard Helm en 1995 escribieron el libro ``Design Patterns: Elements of Reusable Object-Oriented software'', libro que reun�a y clasificaba las soluciones hasta ahora utilizadas. 

El concepto de patr�n se di� a conocer en el libro ``Pattern Language: Towns, Buildings, Construction'' del autor Christopher Alexander, donde se describ�a un lenguaje natural para la construcci�n de edificios. Teniendo el libro anteriormente mencionado como base, fue que estas soluciones a problemas repetitivos fueron nombradas como patrones de dise�o de programaci�n.

Con el tiempo, estas cuatro personas pasaron a llamarse Gang of Four (Banda de los cuatro) y a su vez el nombre del libro paso a ser ``The GOF book''.

\subsubsection{Tipos de patrones}
En total el libro recoge 23 patrones, divididos en tres categor�as seg�n su intenci�n:
\begin{itemize}
	\item Patrones Creacionales:
	\begin{itemize}
		\item Provee mecanismos para la creaci�n de objetos lo que incrementa la flexibilidad y la reutilizaci�n de c�digo existente.
	\end{itemize}
	\item Patrones Estructurales:
	\begin{itemize}
		\item Explica como ensamblar objetos y clases dentro de largas estructuras, mientras que la estructura se mantiene flexible y eficiente.
	\end{itemize}
	\item Patrones de Comportamiento:
	\begin{itemize}
		\item Se encarga de la comunicaci�n eficiente y la asignaci�n de responsabilidades entre los objetos.
	\end{itemize}
\end{itemize}

\subsection{Metodolog�as para el desarrollo de software:}

Los softwares han sido parte de la vida cotidiana de la humanidad desde hace muchos a�os, pero su desarrollo comenz� de una manera muy desorganizada, ya que se basaba en actividades de codificar y arreglar los errores. Los softwares en su mayor�a eran creados sin seguir una l�nea de actividades, por lo que la estructura de estos pod�a variar frecuentemente. Pero como todo en la vida, llega un momento en que hay que organizar la forma en que se hacen las cosas, por lo que surgi� una alternativa llamada metodolog�a. Esta dictaba un proceso bien estructurado para la creaci�n de software, lo cual hac�a el desarrollo de estos m�s predictibles y eficientes.\\

Primero surgieron las metodolog�as tradicionales poseedoras de un plan de trabajo extenso, debido a la documentaci�n de los requerimientos del software, seguidos por la especificaci�n de la arquitectura y una representaci�n de alto nivel del software a desarrollar. Debido a la cantidad de trabajo a realizar, las metodolog�as tradicionales pasaron a ser conocidas como metodolog�as pesadas. Mayormente se utilizaban en software con un alto impacto, ya sea para la vida o la sociedad. Pero los proyectos que no pose�an un impacto tan notable, tambi�n deb�an utilizar las metodolog�as pesadas, por lo que el trabajo era muy lento y lleno de dificultades.\\

En respuesta al trabajo excesivo y minucioso en proyectos sin grandes repercusiones en la sociedad, nacieron las metodolog�as �giles, que fueron destinadas al desarrollo de software mediante la interacci�n con el cliente. Cambios en la estructura del proyecto de forma m�s seguida y un desarrollo r�pido, eran las caracter�sticas que m�s difer�an del prop�sito principal de las metodolog�as pesadas \citep{Awad2005}.\\


Dentro de las metodolog�as �giles se encuentran:

	\paragraph{ \ac{xp}:} Se caracteriza por los ciclos de desarrollo cortos, el incremento de los planes para el desarrollo del software, adem�s de la retroalimentaci�n que se establece con el cliente.
	\paragraph{ SCRUM:}
	 Describe la forma en que los miembros de equipo deben trabajar para poder obtener un sistema flexible en un entorno que var�a de manera constante.
 \paragraph{Agile Lite:}
		``(...)puede ser aplicada a cualquier proyecto, asumiendo que el trabajo a realizar se pueda dividir en peque�as acciones.'' \citep{AgileLiteHumanos}
		Utiliza ciclos de desarrollos cortos.\\


Con la ayuda de la informaci�n anteriormente expuesta, ya es hora de seleccionar una metodolog�a para el desarrollo del \cdis, pero antes, se deben poner en una balanza las caracter�sticas del mismo:


%	TABLE
\begin{table}[h]
	\centering
	\begin{tabular}{|l|l|l|}
		\hline
		\theader
		{ } & { �giles} & { Pesadas} \\ \hline
		Acercamiento & X &  \\ \hline
		\stripe 
		Medici�n del objetivo & X &  \\ \hline
		Tama�o del proyecto & X &  \\ \hline
		\stripe 
		Tipo de administraci�n & X &  \\ \hline
		Perspectiva de cambios & X &  \\ \hline
		\stripe 
		Cultura del equipo & X &  \\ \hline
		Documentaci�n & X &  \\ \hline
		\stripe 
		Orientada & X &  \\ \hline
		Ciclos de desarrollo & X &  \\ \hline
		\stripe 
		Dominio de desarrollo &  & X \\ \hline
		Planificaci�n inicial & X &  \\ \hline
		\stripe 
		Retorno de la inversi�n &  & X \\ \hline
		Tama�o del equipo & X &  \\ \hline
		\stripe 
		Total: & 11 & 2 \\ \hline
	\end{tabular}
	\caption{Resultado de la selecci�n de la metodolog�a}
	\label{tab:resultado}
\end{table}

Seg�n la tabla anterior, el software debe ser desarrollado siguiendo las pautas de las metodolog�as �giles, ya que cumple 11 de los 13 aspectos necesarios para decantarse por las mismas.

Dentro de las posibles metodolog�as �giles a seleccionar se encuentran \ac{xp}, Scrum y Agile Lite. Se decide utilizar \ac{xp}.

%	SUBSECTION
\subsubsection{Descripci�n de la metodolog�a \ac{xp}}

``Es una Metodolog�a ligera de desarrollo de aplicaciones que se basa en la simplicidad, la comunicaci�n y la realimentaci�n del c�digo desarrollado''~\citep{VALLADAREZ2016}.

Objetivos de \ac{xp}:
\begin{itemize}
	\item La satisfacci�n del cliente.
	\item Potenciar el trabajo en equipo.
	\item Minimizar el riesgo actuando sobre las variables del proyecto: costo, tiempo, calidad, alcance.
\end{itemize}

caracter�sticas:
\begin{itemize}
	\item Metodolog�a basada en prueba y error para obtener un software que funcione correctamente.
	\item Es orientada hacia quien produce y usa el software.
	\item Reduce el costo de cambio en todas las etapas del ciclo de vida de la aplicaci�n.
	\item Cliente bien definido.
	\item Los requisitos pueden cambiar.
\end{itemize}

\subsubsection{Artefactos de la metodolog�a \ac{xp}}

\paragraph{Historias de usuario}

Las \ac{hu} representan una breve descripci�n del comportamiento del sistema. Se realizan por cada caracter�stica principal del sistema y son utilizadas para cumplir estimaciones de tiempo y el plan de lanzamientos, as� mismo reemplazan un gran documento de requisitos y presiden la creaci�n de las pruebas de aceptaci�n~\citep{VALLADAREZ2016}..

Cada \ac{hu} debe ser lo suficientemente comprensible y delimitada para que los programadores puedan implementarlas en unas semanas.



\paragraph{Tareas de ingenier�a o desarollo}

Una \ac{hu} se descompone en varias tareas de desarrollo, que describen las actividades que se realizaron en cada \ac{hu}, as� mismo las tareas de ingenier�a se vinculan m�s al desarrollador, ya que permite tener un acercamiento con el c�digo~\citep{VALLADAREZ2016}.. 

%Esta puede ser vista en la tabla \ref{tab:ti}.
%
%\begin{table}[h]
%	\centering
%	\begin{tabular}{|l|l|}
%		\hline
%		\rowcolor[HTML]{C4BBD7} 
%		\multicolumn{2}{|c|}{\cellcolor[HTML]{C4BBD7}{Tarea de Ingenier�a}} \\ \hline
%		N�mero de tarea:  Identificador de la tarea. & \begin{tabular}[|c|]{@{}l@{}}  N�mero de historia: N�mero asignado \\ de la historia correspondiente. \end{tabular}\\ \hline
%		\stripe 
%		\multicolumn{2}{|l|}{\cellcolor[HTML]{EDEBF1}Nombre de Tarea: Describe de manera general dicha tarea.} \\ \hline
%		Tipo de tarea: Tipo al que corresponde dicha tarea. & \begin{tabular}[|c|]{@{}l@{}} Puntos estimados: N�meros de d�as necesarios \\ para desarrollar dicha tarea. \end{tabular} \\ \hline
%		\stripe 
%		Fecha Inicio: Fecha inicial de la creaci�n de dicha tarea. & Fecha Fin: Fecha de la culminaci�n de dicha tarea. \\ \hline
%		\multicolumn{2}{|l|}{Programador Responsable: Nombre del programador a cargo de desarrollar la tarea.} \\ \hline
%		\stripe 
%		\multicolumn{2}{|l|}{\cellcolor[HTML]{EDEBF1}Descripci�n: Informaci�n detallada de dicha tarea.} \\ \hline
%	\end{tabular}
%	\caption{Plantilla: Tarea de Ingenier�a.}
%	\label{tab:ti}
%\end{table}



\paragraph{Pruebas de aceptaci�n}

Antes de conocer en qu� consisten las pruebas de aceptaci�n, es necesario conocer los dos tipos de pruebas que  existen, hoy en d�a, en la industria del software~\citep{Nidhra2012}:

\begin{itemize}
	\item \textbf{Caja negra}: Son las pruebas basadas en los requerimientos especificados y no es necesario examinar el c�digo.
	
	\item \textbf{Caja blanca}: Prueba aplicable �nicamente sobre el c�digo perteneciente a un software desarrollado. Son dise�ados desde el punto de vista del desarrollador. Es principalmente usado para detectar errores l�gicos en el c�digo de un programa. 
\end{itemize}

Las pruebas de aceptaci�n pertenecen a la categor�a de pruebas de caja negra y son de vital importancia para el �xito de una iteraci�n y el comienzo de la siguiente, con lo cual el cliente puede conocer el avance en el desarrollo del sistema y a los programadores lo que les resta por hacer. Adem�s, permite una retroalimentaci�n para el desarrollo de las pr�ximas historias de usuarios a ser entregadas. Estas son com�nmente llamadas pruebas del cliente, por lo que las realiza el encargado de verificar si las historias de usuarios de cada iteraci�n cumplen con la funcionalidad esperada~\citep{XPACT}. 
%
%Esta plantilla puede ser vista en la tabla \ref{tab:pa}.
%% Please add the following required packages to your document preamble:
%% \usepackage[table,xcdraw]{xcolor}
%% If you use beamer only pass "xcolor=table" option, i.e. \documentclass[xcolor=table]{beamer}
%\begin{table}[h]
%	\centering
%	\begin{tabular}{|l|l|}
%		\hline
%		\multicolumn{2}{|c|}{\cellcolor[HTML]{C4BBD7}Prueba de aceptaci�n} \\ \hline
%		\begin{tabular}[c]{@{}l@{}}C�digo: N� �nico, permite identificar \\ la prueba de aceptaci�n.\end{tabular} & \begin{tabular}[c]{@{}l@{}}N� Historia de Usuario: N�mero �nico \\ que identifica a la historia de usuario.\end{tabular} \\ \hline
%		\multicolumn{2}{|l|}{\cellcolor[HTML]{EDEBF1}\begin{tabular}[c]{@{}l@{}}Historia de Usuario: Nombre que indica de manera general \\ la descripci�n de la historia de usuario.\end{tabular}} \\ \hline
%		\multicolumn{2}{|l|}{\begin{tabular}[c]{@{}l@{}}Condiciones de Ejecuci�n: Condiciones previas que deben \\ cumplirse para realizar la prueba de aceptaci�n.\end{tabular}} \\ \hline
%		\multicolumn{2}{|l|}{\cellcolor[HTML]{EDEBF1}\begin{tabular}[c]{@{}l@{}}Entrada/Pasos de Ejecuci�n: Pasos que siguen los usuarios \\ para probar la funcionalidad de la historia de usuario.\end{tabular}} \\ \hline
%		\multicolumn{2}{|l|}{\begin{tabular}[c]{@{}l@{}}Resultado Esperado: Respuesta del sistema que el cliente espera, \\ despu�s de haber ejecutado una funcionalidad\end{tabular}} \\ \hline
%		\multicolumn{2}{|l|}{\cellcolor[HTML]{EDEBF1}\begin{tabular}[c]{@{}l@{}}Evaluaci�n de la Prueba: Nivel de satisfacci�n del cliente sobre \\ la respuesta del sistema. Los niveles son: Aprobada y No Aprobada.\end{tabular}} \\ \hline
%	\end{tabular}
%	\caption{Pruebas de aceptaci�n}
%	\label{tab:pa}
%\end{table}

\paragraph{Tarjetas \ac{crc}}

Las Tarjetas \ac{crc}, permiten conocer que clases componen el sistema y cuales interact�an entre s�.
%
%\begin{table}[h]
%	\centering
%	\begin{tabular}{|l|l|}
%		\multicolumn{2}{c}{\cellcolor[HTML]{C4BBD7}{Tarjeta CRC}} \\\hline
%		Nombre de la clase & \begin{tabular}[c]{@{}l@{}}N�mero de Historia: N�mero\\   de la historia de usuario correspondiente.\end{tabular} \\\hline
%		\multicolumn{2}{l}{\cellcolor[HTML]{EDEBF1}Responsabilidades: Atributos y operaciones de la clase.} \\ \hline
%	\end{tabular} 
%	\caption{Plantilla: Tarjeta CRC}
%	\label{tab:crc}
%\end{table}

\subsubsection{Fases de la metodolog�a \ac{xp}}

\begin{figure}[h]
	\centering
	\includegraphics{xp.png}
	\caption{Fases de XP ~\citep{VALLADAREZ2016}.}
	\label{fig:xp}
\end{figure}

\begin{itemize}
	\item \textbf{Planificaci�n:} La Metodolog�a \ac{xp} plantea la planificaci�n como un di�logo continuo entre las partes involucradas en el proyecto, incluyendo al cliente, a los programadores y a los coordinadores. El proyecto comienza recopilando las historias de usuarios, las que constituyen a los tradicionales casos de uso. Una vez obtenidas estas historias de usuarios, los programadores eval�an r�pidamente el tiempo de desarrollo de cada una, determinando as� el tiempo total de desarrollo del software.
	\item \textbf{Dise�o}: Especificaci�n de c�mo debe ser el Dise�o final de la aplicaci�n, haciendo �nfasis en que un Dise�o sencillo es m�s f�cil de implementar que uno complejo.
	\item \textbf{Codificaci�n}: Implementaci�n de los c�digos necesarios para satisfacer las historias de usuario.
	\item \textbf{Pruebas}: Una vez terminada la codificaci�n, se deben realizar las pruebas pertinentes a la misma.
\end{itemize}

\section*{Conclusiones del cap�tulo}
\begin{itemize}
    \item El estudio de los referentes te�ricos-metodol�gicos asociados a la gesti�n de procesos disciplinarios sirvi� de sustento al \cdis.
    \item El an�lisis de los sistemas hom�logos nacionales e internacionales permiti� definir aspectos importantes del dise�o e implementaci�n tales como artefactos ingenieriles de apoyo, tecnolog�as y herramientas; y de la investigaci�n como la documentaci�n a consultar.
    \item La selecci�n de la metodolog�a \ac{xp}, junto a las herramientas y tecnolog�as, definieron el ambiente de desarrollo que guiar� el desarrollo del \cdis.
\end{itemize}
%	PATRONES DE DISE�O
