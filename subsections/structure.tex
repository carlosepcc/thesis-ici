\section*{Estructura}
El presente trabajo de diploma cuenta con la siguiente estructura: Introducci�n, tres cap�tulos, conclusiones, recomendaciones, glosario y, por �ltimo, referencias bibliogr�ficas. A continuaci�n, se presenta un resumen de las diferentes tem�ticas que se abordan en los cap�tulos.
\subsection*{Estructura capitular}

\paragraph{Cap�tulo 1: Fundamentaci�n Te�rica:} En este cap�tulo se exponen los elementos te�ricos utilizados en la investigaci�n, describiendo adem�s las tecnolog�as, metodolog�as, herramientas y el lenguaje de programaci�n utilizado en el desarrollo de la soluci�n, as� como los principales conceptos involucrados, para una mejor comprensi�n.
\paragraph{Cap�tulo 2: Propuesta de soluci�n:} En este cap�tulo se exponen los elementos que permiten describir la propuesta de soluci�n, tales como requerimientos no funcionales e historias de usuario, para lograr un entendimiento claro del sistema a desarrollar. As� como el modelado de dise�o de las funcionalidades y la implementaci�n de la propuesta de soluci�n, donde se obtiene un recurso inform�tico que brinda los servicios requeridos por el cliente.
\paragraph{Cap�tulo 3: Validaci�n de la soluci�n:} En este cap�tulo se describen las pruebas realizadas al sistema, para comprobar el correcto funcionamiento de la propuesta de soluci�n.

