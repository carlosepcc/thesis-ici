\subsection{Pruebas de aceptaci�n}
Las pruebas de aceptaci�n son definidas por el usuario del sistema y preparadas por el equipo de desarrollo, aunque la ejecuci�n y aprobaci�n final corresponden al usuario. Estas pruebas van dirigidas a comprobar que el sistema cumple los requisitos de funcionamiento esperado, recogidos en el cat�logo de requisitos y en los criterios de aceptaci�n del sistema de informaci�n, y conseguir as�? la aceptaci�n final del sistema por parte del usuario \citep{gutierrez2006pruebas}.
A continuaci�n se muestran algunas de las pruebas de aceptaci�n realizadas. 

\begin{acceptancetest}[t:expresult] % label in brackets
   \testcasecode{HU1\_P1}
   \testcasedescription{Prueba que s�lo se puede acceder a las funcionalidades del sistema si se ha realizado una autenticaci�n exitosa primero}
   \testcaseexeccond{No existen datos de un usuario autenticado en el sistema.}
   \testcaseexecstep{
   \begin{enumerate}
   \item Se ingresan los credenciales correctos
   \item Se ejecuta el m�todo de autenticaci�n
   \item Se muestran los elementos de la interfaz que permiten acceder a las funcionalidades del sistema.
   \end{enumerate}
   }
   \testcaseexpresult{El usuario queda autenticado en el sistema si ingresan los credenciales correctos solamente.}
   \testcasename{Autenticar usuario}
   \testcaseuserstory{1}
\end{acceptancetest}