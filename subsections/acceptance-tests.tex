\subsection{Pruebas de aceptaci�n}
Las pruebas de aceptaci�n son definidas por el usuario del sistema y preparadas por el equipo de desarrollo, aunque la ejecuci�n y aprobaci�n final corresponden al usuario. Estas pruebas van dirigidas a comprobar que el sistema cumple los requisitos de funcionamiento esperado, recogidos en el cat�logo de requisitos y en los criterios de aceptaci�n del sistema de informaci�n, y conseguir as�? la aceptaci�n final del sistema por parte del usuario \citep{gutierrez2006pruebas}.
A continuaci�n se muestran algunas de las pruebas de aceptaci�n realizadas. 

\begin{acceptancetest}[t:hu1p1] % label in brackets
   \testcasecode{HU1\_P1}
   \testcasedescription{Prueba para la funcionalidad: Autenticar usuario. Prueba que s�lo se puede acceder a las funcionalidades del sistema si se ha realizado una autenticaci�n exitosa primero}
   \testcaseexeccond{El usuario no est� autenticado en el sistema.}
   \testcaseexecstep{
   \begin{enumerate}
   \item Se navega a la p�gina donde se encuentra el formulario de autenticaci�n.
   \item Se ingresan los credenciales correctos en el formulario de autenticaci�n.
   \item Se inicia la autenticaci�n a trav�z del bot�n de acci�n del formulario o la tecla Enter.
   \item Se recarga la p�gina.
   \end{enumerate}
   }
   \testcaseexpresult{El usuario queda autenticado en el sistema si se ingresan los credenciales correctos solamente. Se muestran los elementos de la interfaz que permiten acceder a las funcionalidades del sistema y la informaci�n del usuario autenticado justo despu�s de culminar el proceso de autenticaci�n.\\
      Evaluaci�n de la prueba: Satisfactoria}
   \testcasename{Autenticar usuario}
   \testcaseuserstory{1}
\end{acceptancetest}

\begin{acceptancetest}[t:hu1p2] % label in brackets
   \testcasecode{HU1\_P2}
   \testcasedescription{Prueba para la funcionalidad: Autenticar usuario. Prueba que el usuario puede cerrar una sesi�n iniciada.}
   \testcaseexeccond{El usuario est� autenticado en el sistema.}
   \testcaseexecstep{
   \begin{enumerate}
   \item Se inicia el cierre de sesi�n desde el men� del cliente.
   \item Se confirma la acci�n en un cuadro de di�logo.
   \item Se recarga la p�gina.
   \end{enumerate}
   }
   \testcaseexpresult{El usuario deja de estar autenticado en el sistema.\\
   Evaluaci�n de la prueba: Satisfactoria}
   \testcasename{Autenticar usuario}
   \testcaseuserstory{1}
\end{acceptancetest}