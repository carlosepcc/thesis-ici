\subsection{Patrones de dise�o: }

Seg�n el libro Dive Into Design Patterns, los patrones de dise�o son:
\begin{quote}
	``Soluciones t�picas a problemas comunes en el desarrollo de software. Se podr�a decir que son como planos predefinidos que pueden ser adaptados para resolver problemas en el dise�o de la codificaci�n de un programa~\citep{Shevts2019}.''
\end{quote}

Normalmente se confunden con algoritmos, porque ambos conceptos describen soluciones t�picas a problemas conocidos, mientras que un algoritmo describe una serie de pasos a seguir para lograr un objetivo, un patr�n es una descripci�n de alto nivel de la soluci�n, o sea, la codificaci�n de un mismo patr�n puede ser diferente en programas distintos~\citep{Shevts2019}.

Los patrones de dise�o difieren entre ellos debido a su complejidad, el nivel de detalles necesarios y la escala del sistema que se va a implementar. Los patrones de bajo nivel son llamados idiomas y usualmente solo se aplican a un lenguaje de programaci�n. Mientras que los patrones m�s universales y de m�s alto nivel, son llamados patrones arquitect�nicos. Estos �ltimos pueden ser usados en cualquier programa independiente del lenguaje en que sea programado y adem�s pueden ser utilizados para crear la arquitectura completa de un software~\citep{Shevts2019}. 

\subsubsection{Surgimiento de los patrones:}

En un principio, no fueron llamados patrones, ni estaban agrupados, sino que fueron soluciones que se repitieron una y otra vez en el desarrollo de software. Debido a esto, Erich Gamma, Jhon Vlissides, Ralph Johnson y Richard Helm en 1995 escribieron el libro ``Design Patterns: Elements of Reusable Object-Oriented software'', libro que reun�a y clasificaba las soluciones hasta ahora utilizadas. 

El concepto de patr�n se di� a conocer en el libro ``Pattern Language: Towns, Buildings, Construction'' del autor Christopher Alexander, donde se describ�a un lenguaje natural para la construcci�n de edificios. Teniendo el libro anteriormente mencionado como base, fue que estas soluciones a problemas repetitivos fueron nombradas como patrones de dise�o de programaci�n.

Con el tiempo, estas cuatro personas pasaron a llamarse Gang of Four (Banda de los cuatro) y a su vez el nombre del libro paso a ser ``The GOF book''.

\subsubsection{Tipos de patrones}
En total el libro recoge 23 patrones, divididos en tres categor�as seg�n su intenci�n:
\begin{itemize}
	\item Patrones Creacionales:
	\begin{itemize}
		\item Provee mecanismos para la creaci�n de objetos lo que incrementa la flexibilidad y la reutilizaci�n de c�digo existente.
	\end{itemize}
	\item Patrones Estructurales:
	\begin{itemize}
		\item Explica como ensamblar objetos y clases dentro de largas estructuras, mientras que la estructura se mantiene flexible y eficiente.
	\end{itemize}
	\item Patrones de Comportamiento:
	\begin{itemize}
		\item Se encarga de la comunicaci�n eficiente y la asignaci�n de responsabilidades entre los objetos.
	\end{itemize}
\end{itemize}

